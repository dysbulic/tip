\chapter{What is Mentor? (Connie)}

\begin{authors}
	Dr.\ Connie Hood
\end{authors}

\vspace{0.5cm}

\section{Your Introduction to Mentor}

Usually I invite a student into my office for a friendly chat. I will
have noticed that the student is already a Seeker, asking intelligent
questions, unafraid to question the teacher's authority, interested in
big ideas, reading and discussing materials outside of those required
for class. While we chat, I double check how comfortable the student
is in talking with me, how mature the reasoning processes are, how
well the student handles a variety of religions and value patterns.

Nowadays the older mentor students and several teachers help me choose
the students who will be invited to join Mentor. I interview each
student before the invitation is made, to be sure the student is able
to understand the rigors of introspective study and to ensure that the
student understands the rules:

Mentor students must
\begin{enumerate}
	\item Attend Weekly meetings of the group
	\item Read the email in the group list serv and write at least one
journal a week. More is better.
	\item Meditate at least 20 minutes a day. Formal meditation.
	\item Read the books and see the movies in the Mentor list and discuss in
journals and groups
	\item Keep what goes on in the meetings and conversations confidential
\end{enumerate}

In addition to these commitments we ask students to attend one More to
Life Weekend sponsored by the Kairos Foundation, and to take Melodie
Wade's HON 2000-level self esteem and relationship courses. They are
very important for your growth. Most students like to go back to MTL
and be on team, to continue to get more out of that approach to the
Work. (see below for more about Kairos Foundation). The student can
drop out at any time. Of he does drop, we ask that he honor the
secrecy agreement.

\section{The Importance of Having a Teacher}

include stuff from Rumi and Kornfield

\section{Insights and Epiphanies}

\section{The Levels of Mentor}



\section{Mentor Reading List---Level 1: Introductory Materials}

%\subsection{Books}

\begin{sidewaystable*}[htb]
	\centering
	\begin{tabular}{|ll|l|l|}
		\hline
		\multicolumn{2}{|c|}{Books} & \multicolumn{2}{|c|}{Movies} \\
		\hline\hline
		\multicolumn{1}{|c}{Author} & \multicolumn{1}{c|}{Title} & \multicolumn{1}{c|}{In English} & \multicolumn{1}{c|}{Foreign Films} \\
		\hline
		Hesse 					& \emph{Siddhartha} 											& The Razor's Edge							& The Seventh Seal \\
		Richo 					& \emph{How to Be an Adult} 							& Gandhi 												& Antonia's Line \\
		Thich Naht Hanh & \emph{Being Peace} 											& My Dinner with Andre 					& Kundun \\
		Pirsig 					& \emph{Zen and the Art Of Motorcycle Maintenance} & Passage to India 		& Osama \\
		DiSanto \& Steele & \emph{Guidebook to ZMM} 							& Ordinary People 							& Himalaya \\
		Peck 						& \emph{The Road Less Traveled} 					& On Golden Pond 								& Zorba the Greek \\
		Pearson 				& \emph{Awakening the Heroes Within} 			& The Last Temptation of Christ & Muriel's Feast \\
		Beattie 				& \emph{Codependent No More} 							& Kinsey 												& Crouching Tiger, Hidden Dragon \\
		Paulus 					& \emph{Hope for the Flowers} 						& American Beauty 							& Ararat \\
		de Saint-Exup\'ery & \emph{The Little Prince} 						& Pleasantville 								& Tigers Can Fly \\
		----- 					& \emph{The Giving Tree} 									& Schindler's List 							& Story of the Weeping Camel \\
		Frankl 					& \emph{Man's Search for Meaning} 				& Hotel Rwanda 									& Raise the Red Lantern \\
		Ferguson 				& \emph{The Aquarian Conspiracy} 					& Empire of the Sunday 					& Rashomon \\
		Hoff 						& \emph{The Tao of Pooh} 									& The Last Emperor 							& Alexander Nevsky \\
		----- 					& \emph{The Te of Piglet} 								& The Lord of the Rings 				& Ivan the Terrible (pts. 1 \& 2) \\
		Rogers 					& \emph{A Way of Being} 									& Narnia 												& Battleship Potempkin \\
		de Mello 				& \emph{Awareness} 												& Shadowlands										& Virgin Spring \\
		Campbell 				& \emph{An Open Life} 										& Breaking the Waves 						& Through a Glass Darkly \\
		Armstrong 			& \emph{A History of God} 								& Tora Tora Tora 								&  \\
		Kornfield 			& \emph{A Path with Heart} 								& The Longest Day 							&  \\
		Helminski 			& \emph{Living/Presence} 									& Amadeus 											&  \\
		Thich Naht Hanh  & \emph{Living Buddha, Living Christ} 		& Les Miserables (musical) 			&  \\
		Huston Smith 		& \emph{The Illustrated World Religions} 	& Man of La Mancha (musical) 		&  \\
		Maslow 					& \emph{Religions, Values, and Peak Experiences} & Rabbit-Proof Fence 		&  \\
		----- 					& \emph{Further Reaches of Human Nature} 	& Snow Walker 									&  \\
		Cassirer 				& \emph{Essay on Man} 										& Whale Rider 									&  \\
		Plato 					& \emph{Republic} (Cornford Edition) 			& Fantasia (that Disney thing) 	&  \\
		----- 					& \emph{Apology} 													& Song of the South (uncut; again Disney) &  \\
		----- 					& \emph{Crito} 														&  &  \\
		----- 					& \emph{Phaedo} 													&  &  \\
		----- 					& \emph{Phaedrus} 												&  &  \\
		Einstein 				& \emph{Ideas and Opinions} 							&  &  \\
		Jung 						& \emph{Psyche and Symbol} 								&  &  \\
		\hline
	\end{tabular}
	\caption{Level 1 Mentor Reading List}
	\label{tab:L1readinglist}
\end{sidewaystable*}

\section{A Brief History of Mentor}

51234

\section{To throw in this chapter somewhere}

\subsubsection{Journal: Nathan Payne, 7 August 2007}

I have formed a new definition of mentor. That is, after assuming that there is a definition, that that definition can be quantified, and that verbalizing it in English accurately portrays it. This is my simplified definition after a long tangent of deduction. 

And it is: ``To protect us from suicide.''

\bigskip

(After numerous requests for context, Nathan elaborated.)

\subsubsection{Journal: Nathan Payne, 7 August 2007}

I realize now this requires a bit of redefining and clarifying. By suicide I don't refer to literal suicide. It closer to when Ralph Waldo Emerson said: ``Imitation is suicide.'' Some exercise definitions of mentor have been ``to learn to love everyone,'' ``improve yourself,'' ``higher quality of life,'' etc. But all of these things are not the central goal (e.g., you must learn to love yourself before you can love the world, likewise the first goal would appear to be to love yourself). The question comes to mind, what is it explicitly, precisely, that mentor tries to do initially? What does mentor do that can sum up all of these different definitions that are true in their own context?
 
Without looking at the history of the word mentor, without looking at the history of the program, I simply looked at Connie who started all of this. To the best that I understand her, dispite all her years of broad research and thick library's, What really gave her satisfaction was when she sat down and worked with suicidal kids. To extend her heart and help those in pain by first listening to their pain and then helping them see the solution to their own problems.
 
Does her helping of countless suicidal individuals transcend into her helping of countless names though mentor? Does the passion for one go into the passion of another? I don't like sitting here and making numerous assumptions and projections, but I see a connection.
 
This derives out of doing some shadow work for myself, and a realization came to me that although I've never taken a knife to my skin, I've let myself die multiple times. I've dragged my feet, avoided pain, dulled my senses, intellectualized problems, lied to myself, blinded myself. The phrase from the movie Farenheight 451--- ``We're not living, we're all just passing time''--- is another way to say it.
 
So what is Mentor? Coming in I heard several times that it was referred to ``deconditioning.'' Deconditioning from what? Deconditioning from all the webs and foundations and assumptions established though society, schools, culture, religion and have you. Is this true so far? And this is done so that we may think clearer, yes? So that we may ``wake up,'' be ``reincarnated.''
 
``The masses of men lead lives of quiet desperation.'' - Thoreau
 
And what where we before this, asleep? dead? and if we where dead, even though you may be surrounded by people around you who throw you answers and wisdom, it is you and only you who can do the work and pick yourself up. You can point fingers and blame, but that docent help you beginning to solve your own problems.
 
So instead of doing one of my long and hard to follow rants :P I realized a simple phrase could describe it.
 
What is mentor? Or more specifically, whats one way to phrase what mentor means to me?
To save myself from suicide.

\subsubsection{Journal: Eric Hoy, 3 September 2007}

The meeting tonight was just plain awesome. I loved the personal sharing that occurred. Add pain and issues to the pot, stir well, and make one batch of caring and love. Now to boundaries.

My boundaries are the sharpest when it comes to emotional pain. I have strong boundaries (of the neurotic type) to try and protect myself from pain. The pain I fear most is pain that comes from rejection. I find that pain to be intolerable. It just reminds me too much of the way things used to be. My boundaries here are highly neurotic and general cause all sorts of counterproductive behavior. I notice that mainly it causes a strange tension between myself and another person and generally results in one way conversations were either myself or the other person does all of the talking. When I have to do the talking, I feel that the relationship is forced and fake that I am only maintaining it for my own benefit. That is how I have often felt with my early friends. In fact, almost all of my friends before Mentor. I had to force a relationship as no one would be my friend otherwise. Well, no one I would let be my friend. My perfectionism killed me there. I have to have friends who I saw as "better" than me or they wouldn't remain my friends for long. I had to have someone to look up to. I purposefully put myself in the powerless position that way I would not have to be responsible or brave. I would not have to take charge or be under pressure. Power is the major boundary here. If I feel powerless, my controls kick in hard to rectify the situation, sometimes violently. I have more than once physically attacked my sister when she belittled me. Not in several years, and I'm not proud of it. My anger would just carry me away. The lack of control and power outside my family made me demand it in my family. If I don't deal with this, I think I could really hurt someone in a marriage or position of power. To be honest, despite what I think, I would be easily tempted by power. Without mentor, I would have gone power mad if I ever gained power that my Ego could not deny. In each longing for power there is a cry for help. Help for pain that never healed. A lack of power that is really felt. A lack of control. I seek control and power to prevent more pain. I figure that more defenses mean less pain, and I haven't found the courage to accept the pain on a deep level so I seek control to offset the pain to a level I can take. I go as far as to repress my emotions to avoid feeling the pain. The Anima guards the chest. 

\section{Quotes}

\section{Works Cited}

\section{Further Reading}


\bibliographystyle{mla}		% the style you want to use for references.
\bibliography{mr,refs}				% the files containing all the articles and books you ever referenced.
