\chapter{Mindfulness (Craig)}

\begin{authors}
	Craig Schuff
\end{authors}

\section{Introduction}

Definition of Mindfulness\index{mindfulness}:
\begin{enumerate}
	\item Observing in the present moment
	\item Leaving off judgment
\end{enumerate}

\section{Noticing}

\begin{enumerate}
	\item What we do not notice still occurs
	\item When we notice we can then act on this thing as it has been acting on us
\end{enumerate}

\section{Naming}

\begin{enumerate}
	\item After noticing something new we seek to attach a name to it.  What is this new thing?
	\item The names we give identify what we are doing in the simplest terms available
	\item If a better name becomes apparent then use it
	\item Naming will greatly improve our practice of noticing. 
\end{enumerate}

\section{Patterns}

\begin{enumerate}
	\item With a little time spent noticing and naming we start to see things repeating.
	\item This is in itself something new to notice.  "I have done this before!"
\end{enumerate}

\section{Mindfulness Practices}

\begin{enumerate}
	\item Mindfulness: Observing in the present moment
	\item Practice:  You get better at it!
\end{enumerate}

\subsection{Senses}

Notice and name your senses. 

Sight

Sound

Touch

Smell

Taste 

Part 1 

Start with an experience.  Leaves in the fall for example. 

As you experience a sensation pull out the part that is sight.  Explore that sensation.  Roll it around in your mind while repeating "Sight... Sight... Sight," in your head. 

Pull out the sound.  Explore it.  Leaves rustling "Sound... Sound... Sound." 

Pull out the touch.  Wind blowing.  Wind in hair.  Air over skin.  Chill.  Watering eyes.   "Touch... Touch... Touch" 

Pull out the smell.   

Pull out the taste 

Not every experience will involve strong sensations in every category.  No need to create what is not there.  Now, if the experience involved apple pie, ah! 

Part 2 

Leaves again 

Within an experience take each sense again.  This time name what other sensations directly interact with this one. 

"Sight... Sight... Sight"

The feeling of my eyes during sight "Touch" 

"Sound... Sound"

Feeling my ears during sound "Touch"

Feeling sharp sound on my ears "Touch"

Feeling a loud low sound on my body "Touch" 
 

Wind on my face "Touch"

Wind moving the leaves "Sight"

Pressure in my ears "Touch"

crunching noise "Sound"

Person walking through leaves "Sight"

\subsection{Colors}

First pick a color.  Everywhere you go notice this color.  When you see your color say it in your mind.  "Red"... "Red"... "Red."  Do this whatever else you are doing, where ever you are.  During conversations, during lectures, while walking, while writing notice your color and say it in your mind.  
 
Doing this for any length of time is where discipline comes in.  You will be distracted, you will forget to notice your color despite all of your best intentions.  When you notice that you are not doing the exercise recognize it and start again.  
 
Wear a rubber band on your wrist.  You will find often that you have forgotten to see your color.  The rubber band will help remind you.  It is easy to go days without remembering color, it is harder to miss a rubber band on your wrist.  
 
Each day you may pick a new color to notice.  Choose something other than black or white. 
 
 
 
Journal during this experience.   
What do you notice?   
What do you notice yourself doing? 
Does the experience change at all?  Is it different thinking "Red" after 20 minutes than it is after 6 hours?  
Is it easy to remember? 
Is it difficult to do something as simple as naming a color for very long? 
Do you become Frustrated?  Tired?  Bored?  Elated? 
Notice yourself in relationship to this discipline. 
 
 
You do not need to write all of your journal in one sitting.  Write what you notice even if it is only one thought, one sentence, one word.  If you notice more the next day then add more.

\section{Quotes}

\section{Works Cited}

\section{Further Reading}

\bibliographystyle{mla}		% the style you want to use for references.
\bibliography{mr,refs}				% the files containing all the articles and books you ever referenced.
