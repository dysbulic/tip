\chapter{Critical Thinking (Quinton, Eamon)}

\begin{authors}
	Eamon Ryan and Quinton Westrich
\end{authors}

\section{Introduction}

If you're reading this chapter, I think it's safe for me to assume that you have decided to examine yourself and your conception of the world at a deeper level than the \emph{hoi polloi}\index{hoi polloi}, i.e.\ the common man. No doubt, not everyone chooses this path-- what Robert Frost called "the road less traveled." Should the road prove a bit bumpy (often a good sign, a precursor to self-transformation!) we should develop some tools to aid our journey. 

When I started Mentor, I had already spent a few years reading and thinking about philosophical questions. I was eager to join a group of intellectuals in a search for deeper truths. However, I was deeply bothered by the seemingly incessant assumption that there were flaws in my thinking, or that I was somehow a beginner in the search for truth. Grudgingly, I traveled forward, disdainfully examining things I had already "figured out" to humor Dr.\ Hood or to prove the assumption vacuous. 

I can now admit that my reasoning is still loaded with faulty assumptions and poor logic. However, it's a whole world clearer than it was to begin with! And the thought of having an idea "figured out" seems quite presumptuous to me now. My experience has suggested to me that new data will often emerge and force me to change my current beliefs if I wish to remain honest and accurate with my observations. In this sense, humility is not something which only virtuous people have, but something which arises out of necessity and awareness.

Each year, Dr.\ Hood included in her introductory speech to incoming Honors freshman a bit of advice: "Be a little humble about what you \emph{think} you know." Now I realize she wasn't talking down to us as I had first interpreted. She was sharing her wisdom about a way of being. It was not intended solely for freshmen, but as a piece of advice to live with and hold even as we become upper-classmen, graduate students, and professionals. 

With that said, this chapter is essentially about learning how to think. The assumption is that you're probably not exactly right about everything 100\% of the time. Notwithstanding, we proceed with hope of at least improving our attempts at seeking awareness. We'll introduce some tools for making our ideas clear, keeping our reasoning sound, and keeping ourselves honest. The tools include an awareness of both a collection of examples of pseudoreasoning, or \emph{logical fallacies}, and common ways our emotions cloud our judgments, or \emph{defense mechanisms}.

These tools are also useful for evaluating the utility and validity of arguments presented by others. If we each had to figure \emph{everything} out for ourselves, it would take 10 years to just establish that $1+1=2$ \cite{WhRu:62}! In order to speed up the process, we can read the works of others and listen to their arguments \emph{critically}, accepting what we see as likely, reasonable, or attractive and rejecting what we see as unlikely, unreasonable, or repulsive. While all this is good in theory, we cite a journal below of what one mentor student experiences in reality.

\subsubsection{Journal: Eric Hoy, 29 June 2007}

\begin{quotation}
Well I'm back at home and still slightly irritated from the car trip. I hate watching my parents do the same things I do. Especially when I don't like those things such as over-criticism, narrow-mindedness, and the "win" the argument mentality. I realized another reason that I don't like to be criticized or appear at a loss for words. In my family (especially with my dad), silence and loss of words is taken as having "lost" the argument. I noticed this dynamic as I was talking with my dad. I was very tired today and in an argumentative mood so I saw how I can argue. I can contradict myself 10 or 20 times in an argument. I don't get the facts straight once the discussion gets heated. I noticed my dad uses name calling to derail one's argument (calling me a "liberal," "stuck up," or "self-centered" as if being these ruin my argument). I have fallen into this logical fallacy many times when I feel backed into a corner (like him), but never to the extent he does. Saying the "wrong" thing ends the discussion. He treats the arguments like games (another trap for me). I don't think they're so fun anymore. It is his way of interaction (and it drives my mom up the wall sometimes). I do see the two in one mind when they are criticizing the stupidity or recklessness of others (guess where my complexes with security and competence come from, I just realized this). Again this irritated me greatly, another hole I can fall into, over-criticizing. Basically these issues could be summed up as security issues to me. Ways of defending one's competence and beliefs. Being my parents for so long is not as bad as I think it is.

Erik
\end{quotation}


\section{What is critical thinking?}

\subsection{Claims, Meaning, and Truth}

In this section, we'll present some definitions and motivation for pursuing our as-of-yet undefined term "critical thinking." Let us remedy the undefinedness; but, first, we need a leading definition: A \bigwerd{claim} is a statement for which the labels `true' and `false' make gramatical sense. So, not everything that is said is a claim: interogative, exclamatory, and imperative statements are excluded, while declarative statements are generally included. The common purpose of a claim is \emph{to communicate information}. However, this purpose is not usually realized in actual conversation. Humans often engage in sarcasm, small talk, and generally non-informative communication to keep a balance of human needs; we are emotional as well as reasoning creatures. Thus it has become quite tricky to spot a claim. Often we mistake witty comments for claims.

We stop here to remark about the words `true\index{true}' and `false' used above. While the definition given, is often valid, we might come across cases in which we have a claim which defies this characterization. For example, Heraclitus's statement "The way up is the way down," and many Zen koans fit this latter category of claims. These are the subject of Chapter \ref{ch:paradox}, which discusses \bigword{paradox}. In our definition, we allow the case in which a claim is claimed to be \bigword{both true and false} or \bigword{somewhat true} or \emph{somewhat false}. Often these cases can be resolved into a classification of \bigword{relatively true} or \emph{relatively false} with a change in context.

For example, you might say "Rainy weather is depressing." What does this \emph{mean}?\footnote{Admitedly, such an analysis of this particular statement is ridiculous. However, I thought it a neutral example, suitable for demonstration purposes. Nontrivial examples will be cited later.} This leads us to our next caveat: Meaning only exists within some context. This you can verify by experiment. In our example, the \emph{implicit} context is that of the emotional state of all humans, that is, all humans feel depressed whenever it is raining in that particular human's location.

\subsection{Critical Thinking, Dialogue, and Dialectic}

\textbf{Critical thinking}\index{critical thinking} is the careful, deliberate determination of whether we should accept, reject, or suspend judgment about a claim---and of the degree of confidence with which we accept or reject it.

\subsubsection{dialectical}
noun - The art or practice of arriving at the truth by the exchange of logical arguments.
The process especially associated with Hegel of arriving at the truth by stating a thesis, developing a contradictory antithesis, and combining and resolving them into a coherent synthesis.
\subsubsection{sophists}
noun - One skilled in elaborate and devious argumentation. A scholar or thinker.
Sophist Any of a group of professional fifth-century B.C. Greek philosophers and teachers who speculated on theology, metaphysics, and the sciences, and who were later characterized by Plato as superficial manipulators of rhetoric and dialectic.
\subsubsection{didactical}
adjective - intended for instruction; instructive: didactic poetry.
inclined to teach or lecture others too much: a boring, didactic speaker.	teaching or intending to teach a moral lesson. didactics, (used with a singular verb) the art or science of teaching.
adj.Intended to instruct. Morally instructive. Inclined to teach or moralize excessively.

\section{Some Common Logical Fallacies and Pseudoreasoning}

Logical fallacies and pseudoreasoning enrich our lives in ways we might never have anticipated...

\subsubsection{Non Sequitor}
Latin for "it does not follow," ask yourself "Does this logically connect?" In order to be clear in our arguement or discussion, our thought progression should be linked together. 

\subsubsection{Overgeneralization}
Qualifying and asserting a claim as an absolute and not addressing other existing possibities, i.e. asserting a claim holds in a larger context because it applied to a previous specific context.

\subsubsection{Oversimplification}
Simplifying a definition may be usefull while under a constraint of time but with specific language orientation and also scientific processes, breaking down an arguement could create ambiguity in a discussion, debate, or writing. Oversimplication is a very important concept to remember especially when considering language use in dialectical reasoning.

\subsubsection{Post hoc, ergo propter hoc}
Latin for "after this, therefore because of this?." Events sometimes occur in a successive fashion but do not necessarily become an explicit part of reasoning for another event.

\subsubsection{Red herring}
This could also be referred to in mentor terminology as "Bouncing." Example - Having a conversation about G.W. Bush and then talking about Al Gore's sons' arrest for marijuana.

\subsubsection{Reificaiton}

\subsubsection{Relativist fallacy}

\subsubsection{Resorting to Clich\'e}
Using a popular phrase or expression to summate an action or dialectical arguement.

\subsubsection{Sequential fallacy}
reference to check on if different then post hoc ergo propter hoc house of quinton

\subsubsection{Slippery slope}

\subsubsection{Slothful principle}

\subsubsection{Spotlight}

\subsubsection{Straw man}

\subsubsection{Style over substances}
Looks over content.

\subsubsection{Two wrongs make a right}

\subsubsection{Unevaluated contingency}


\section{Defense Mechanisms}

\begin{quote}
	"Anything outside yourself, this you can see and apply your logic to it$\ldots$ but it's a human trait that when we encounter personal problems, those things most deeply personal are most difficult to bring out for our logic to scan. We tend to flounder around, blaming everything but the actual, deep-seated thing that's really chewing on us."
	\begin{flushright}
		in \emph{Dune}
	\end{flushright}
\end{quote}


\subsubsection{Assimilation}

\subsubsection{Denial}

\subsubsection{Displacement}

\subsubsection{Externalization}

\subsubsection{Projection}

\subsubsection{Rationalization}

\subsubsection{Reaction formation}

\section{An Application to Ethics}
\subsection{Ethos, Logos, and Pathos}
\subsection{What is Rhetoric?}
\subsection{What is a Syllogism?}

\section{Clarity...}

\section{Quotes}

\section{Works Cited}


\section{Further Reading}

A full onslaught into the subject of critical thinking can be found in any number of standard texts often used in university philosophy courses. One which has been used as a reference herein is \cite{MoPa:01}. We've also pulled material from the Tennessee Tech \emph{Honors Handbook} \cite{Hon:06} and \emph{Thinking Critically About Ethical Issues} \cite{Ru:04}.

\bibliographystyle{mla}		% the style you want to use for references.
\bibliography{mr,refs}				% the files containing all the articles and books you ever referenced.
