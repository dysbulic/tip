\chapter{Wisdom (Jon)}

\begin{authors}
	Jon Jones
\end{authors}

\section{What is Wisdom?}

\begin{quote}
Where is the Life we have lost in living?\\
Where is the wisdom we have lost in knowledge?\\
Where is the knowledge we have lost in information?"\\
\begin{flushright}
	T.S. Eliot
\end{flushright}
\end{quote}

What is wisdom? What does the word "wisdom" mean? Where are the wise sages and their teachings? To many, wisdom is an indefinable yet invaluable quality that seems to be just beyond succinct description. Some note psychologists, such as Dr. Robert Sternberg and others, have said that one needs a certain level of wisdom to identify and understand wisdom. Furthermore, some believe there is a prerequisite level of knowledge or intelligence needed either to decode the meaning of a wise assertion or to see the potential wisdom in a statement. Can wisdom be taught? The answer may at best be, yes and no. It is perhaps more easily conveyed through an analogy about a well kept and robust garden. In the Garden of wisdom there are kernels of insight and seeds of knowledge. The water could be considered as experience and the soil as the mind with its intelligence and predisposition to novelty. Lastly, we could imagine the sun as patience and think of it as representing the cyclic nature of many aspects of personal growth.

However, we've still not reached a true definition or even understanding of the many meanings and manifestations of wisdom. The analogy is reasonable for starting out, but there is a form of wisdom in knowing the boundaries between multidimensional reality and multifaceted symbolism / metaphor. Is wisdom a way of thinking? Is it a way or method of reasoning, discerning, decision making, or structuring thoughts? Is it the way one prioritizes aspects or actions according to some moral or ethical principle? Is it a way of knowing, objectively or subjectively, the truest nature of things? Is it a way of knowing the most efficient or most practical things and courses of action? Is wisdom the aggregate accretion of esoteric knowledge or truth, long ago discovered and closely guarded? Is wisdom a way of being? Is it a state of being? Is it a state of mind or a state of development? Is it part of a larger process of development? Is it the final stage or the beginning of a whole other paradigm and ontology? Is all wisdom created equal? Is it all the same wisdom or are there many kinds of wisdom?

The word "wisdom" is very much like many other English words in that it is composed of letters, has an etymology,   has separate definitions depending on context, and is a referent (not the thing itself but a symbolic representation of the thing. As a word and a concept, wisdom and its variety of meanings is subject to change. If one wants to solve a mystery, they follow the clues. If one wants to define wisdom, then one must follow the word in its denotations and connotations through the ages. Above all else, the most important aspect of the search must be context. Definitions come from specific contexts, and hence the OED's incredibly large etymologies. Wisdom is perhaps also analogous to a tree, having large dendrite-like branchings fanning out according to time, place, meaning, and use.

As we track wisdom through the ages we see a move from a single wisdom into a more diversified pair that expand upon the original grouping of knowledge, sayings, beliefs, and prudent actions held together under the banner of "good" (if not best) things to do and ways to live). There are then two main branches of wisdom in the ancient societies beginning about 5,000 BCE. These branches are what I call Practical Wisdom and Divine Wisdom. The delineation is as follows.
 
[insert citations and proff of what I'm saying]
 
Practical wisdom came to encompass and perhaps spur the development of such ideas and structures as cultural conventions, societal beliefs about the way the world worked, conventional wisdom about society and environment. Divine wisdom, on the other hand, seems to have been integrated with if not developed out of the teachings of early religion and mythology. Such things as truth from religions revelation, transcendental experiences, paradoxical observations, mystical meanings, and introspective thought in relation to the world and life and death seem to have all come from a related set of experience and thinking.

We must remember that wisdom is much like a family tree. Wisdom and its associated words belonged to these cultures and were in daily use by the people. It grew and changed with the people as they used these concepts and explored them, creating a more divers spectrum of ideas, heuristics, and ways of acting. Volitional, affective, intellectual. And so, wisdom has come to reflect their beliefs as well as their priorities in the writings that remain.
 
[Insert history of wisdom from anthropology books / handbook on wisdom / 1990 wisdom book / holliday and chandler]
 
[talk about the models]
 
[talk about ways it relates to a path of development]
 
[relate that to mentor]
 
you enrich the ming / soil of the garden with mindfullness trainning.
 
it's not just a werstern conception
 
[Nvak's world's wisdom]
 
 
[ our definitins of wisdom and what we think it means and why its important.]
 
[box]
[square]
[rectangle]
[two equivalent triangles stuck together at the hypotenuse]

\section{Sternberg Stuff (scanned)}


\section{Russell's "Knowledge and Wisdom"}

(We need to truncate this! There's a typo in here somewhere!)

Most people would agree that, although our age far
surpasses all previous ages in knowledge, there has
been no correlative increase in wisdom. But agreement
ceases as soon as we attempt to define `wisdom' and
consider means of promoting it. I want to ask first
what wisdom is, and then what can be done to teach it.
There are, I think, several factors that contribute to
wisdom. Of these I should put first a sense of
proportion: the capacity to take account of all the
important factors in a problem and to attach to each
its due weight. This has become more difficult than it
used to be owing to the extent and complexity of the
specialized knowledge required of various kinds of
technicians. Suppose, for example, that you are
engaged in scientific research in medicine. The work
is difficult and is likely to absorb the whole of your
intellectual energy. You have not time to consider the
effect which your discoveries or inventions may have
outside the field of medicine. You succeed (let us
say), as modern medicine has succeeded, in enormously
lowering the infant death-rate, not only in Europe and
America, but also in Asia and Africa. This has the
entirely unintended result of making the food supply
inadequate and lowering the standard of life in the
most populous parts of the world. To take an even more
spectacular example, which is in everybody's mind at
the present time: You study the composistion of the
atom from a disinterested desire for knowledge, and
incidentally place in the hands of powerful lunatics
the means of destroying the human race. In such ways
the pursuit of knowledge may become harmful unless it
is combined with wisdom; and wisdom in the sense of
comprehensive vision is not necessarily present in
specialists in the pursuit of knowledge.

Comprehensiveness alone, however, is not enough to
constitute wisdom. There must be, also, a certain
awareness of the ends of human life. This may be
illustrated by the study of history. Many eminent
historians have done more harm than good because they
viewed facts through the distorting medium of their
own passions. Hegel had a philosophy of history which
did not suffer from any lack of comprehensiveness,
since it started from the earliest times and continued
into an indefinite future. But the chief lesson of
history which he sought to inculcate was that from the
year 400AD down to his own time; Germany had been the
most important nation and the standard-bearer of
progress in the world. Perhaps one could stretch the
comprehensiveness that contitutes wisdom to include
not only intellect but also feeling. It is by no means
uncommon to find men whose knowledge is wide but whose
feelings are narrow. Such men lack what I call wisdom.

It is not only in public ways, but in private life
equally, that wisdom is needed. It is needed in the
choice of ends to be pursued and in emancipation from
personal prejudice. Even an end which would be
noble to pursue, if it were attainable, may be pursued
unwisely if it is inherently impossible of
achievement. Many men in past ages devoted their lives
to a search for the philosopher's stone and the elixir
of life. No doubt, if they could have found them, they
would have conferred great benefits upon mankind, but
as it was their lives were wasted. To descend to less
heroic matters, consider the case of two men, Mr A and
Mr B, who hate each other and, through mutual hatred,
bring each other to destruction. Suppose you go to
Mr A and say, 'Why do you hate Mr B?' He will no doubt
give you an appalling list of Mr B's vices, partly
true, partly false. And now suppose you go to Mr B. He
will give you an exactly similar list of Mr A's vices
with an equal admixture of truth and falsehood.
Suppose you now come back to Mr A and say, 'You will
be surprised to learn that Mr B says the same things
about you as you say about him', and you go to Mr B
and make a similar speech. The first effect, no doubt,
will be to increase their mutual hatred, since each
will be so horrified by the other's injustice. But
perhaps, if you have sufficient patience and
sufficient persuasiveness, you may succeed in
convincing each that the other has only the normal
share of human wickedness, and that their enmity is
harmful to both. If you can do this, you will have
instilled some fragment of wisdom.

I think the essence of wisdom is emancipation, as far
as possible, from the tyranny of the here and now. We
cannot help the egoism of our senses. Sight, sound
and touch are bound up with our own bodies and cannot
be impersonal. Our emotions start similarly from
ourselves. An infant feels hunger or discomfort, and
is unaffected except by his own physical condition.
Gradually with the years, his horizon widens, and, in
proportion as his thoughts and feelings become less
personal and less concerned with his own physical
states, he achieves growing wisdom. This is of course
a matter of degree. No one can view the world with
complete impartiality; and if anyone could, he would
hardly be able to remain alive. But it is possible to
make a continual approach towards impartiality, on the
one hand, by knowing things somewhat remote in time or
space, and on the other hand, by giving to such things
their due weight in our feelings. It is this approach
towards impartiality that constitutes growth in
wisdom.

Can wisdom in this sense be taught? And, if it can,
should the teaching of it be one of the aims of
education? I should answer both these questions in the
affirmative. We are told on Sundays that we should
love our neighbors as ourselves. On the other six days
of the week, we are exhorted to hate. But you will
remember that the precept was exemplified by saying
that the Samaritan was our neighbor. We no longer
have any wish to hate Samaritans and so we are apt to
miss the point of the parable. If you want to get its
point, you should substitute Communist or
anti-Communist, as the case may be, for Samaritan. It
might be objected that it is right to hate those who
do harm. I do not think so. If you hate them, it is
only too likely that you will become equally harmful;
and it is very unlikely that you will induce them to
abandon their evil ways. Hatred of evil is itself a
kind of bondage to evil. The way out is through
understanding, not through hate. I am not advocating
non-resistance. But I am saying that resistance, if it
is to be effective in preventing the spread of evil,
should be combined with the greatest degree of
understanding and the smallest degree of force that is
compatible with the survival of the good things that
we wish to preserve.

It is commonly urged that a point of view such as I
have been advocating is incompatible with vigour in
action. I do not think history bears out this view.
Queen Elizabeth I in England and Henry IV in France
lived in a world where almost everybody was fanatical,
either on the Protestant or on the Catholic side. Both
remained free from the errors of their time and both,
by remaining free, were beneficent and certainly not
ineffective. Abraham Lincoln conducted a great war
without ever departing from what I have called wisdom.

I have said that in some degree wisdom can be taught.
I think that this teaching should have a larger
intellectual element than has been customary in what
has been thought of as moral instruction. I think that
the disastrous results of hatred and narrow-mindedness
to those who feel them can be pointed out incidentally
in the course of giving knowledge. I do not think that
knowledge and morals ought to be too much separated.
It is true that the kind of specialized knowledge
which is required for various kinds of skill has very
little to do with wisdom. But it should be
supplemented in education by wider surveys calculated
to put it in its place in the total of human
activities. Even the best technicians should also be
good citizens; and when I say 'citizens', I mean
citizens of the world and not of this or that sect or
nation. With every increase of knowledge and skill,
wisdom becomes more necessary, for every such increase
augments our capacity of realizing our purposes, and
therefore augments our capacity for evil, if our
purposes are unwise. The world needs wisdom as it has
never needed it before; and if knowledge continues to
increase, the world will need wisdom in the future
even more than it does now.

\begin{table*}
	\centering
	\begin{tabular}{lp{6cm}}
		\hline\hline
		Author(s) & Definition \\
		\hline
		Robinson	& \emph{Three historical definitions}: \\
							& \emph{Greek:} an intellectual, moral, practical life; a life lived in conformity with truth, beauty \\
							& \emph{Christian:} a life lived in pursuit of the divine, absolute truth \\
							& \emph{Contemporary:} a scientific understanding of laws governing matter in motion	\\
		\hline
		Csikszentmihalyi &	word	\\
		and Rathunde						&		word	\\
		\hline
	\end{tabular}
	\caption{Definitions of Wisdom}
\end{table*}

\begin{flushright}
by Bertrand Russell
\end{flushright}

\section{Quotes}

\section{Works Cited}

\section{Further Reading}


\bibliographystyle{mla}		% the style you want to use for references.
\bibliography{mr,refs}				% the files containing all the articles and books you ever referenced.
