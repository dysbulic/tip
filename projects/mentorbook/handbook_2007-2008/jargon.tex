\chapter{Jargon}

\begin{description}
	\item [Bodhisattva] a Buddhist saint; one willing to return to the cave to serve others
	\item [Bounce] v. to giggle; to move consciousness away from the deeply focused subject at hand (Meditation helps focus stay deep.); n. momentary loss in the concentration of an individual or group; comic relief; a quick shift from negative to postive (see 'plus five/minus five')
	\item [Box] v. to keep confidential (e.g., "Box what I'm about to tell you."); n. limited picture; a metaphor referring to the boundaries/limits of one's current world view, e.g., A 'god-box' is how one defines/views god. e.g., 'Breaking boxes' refers to letting go of older worldviews.
	\item [Consciousness] "The state of being conscious; awareness of one's own existence, sensations, thoughts, surroundings, etc." (various authors find this word difficult to define) "Consciousness, we shall find, is reducible to relations between objects, and objects we shall find to be reducible to relations between different states of consciousness; and neither point of view is more nearly ultimate than the other." (T.S. Eliot, doctoral dissertation)
	\item [Cor]	the Latin word for 'heart'
	\item [Deep] dropped consciousness ('go deep' = 'drop your consciousness'); disiplined metacognitve state; When one is able to expand their mind and not judge where it's going; a position where one is able to sift though their subconscious and bring up supressed themes
	\item [Disciplines] "behavior in accord with rules of conduct; behavior and order maintained by training and control" (dictionary.com)
	\item [Dump] to vent, say, or write an irrational emotional tirade, just because you need to say it
	\item [Dump journal] a label put in a subject header of a journal to warn everyone that the writer understands this is a rant (Rants do not invite logical responses.)
	\item [Edge] working on your edge ???
	\item [Ego] that part of the consciousness that we identify as "I" (Freud uses "ego" to mean the healthy, well-balanced adult mind, but Buddhists speak of the Ego-mind as that part of the consciousness that is fear-driven and neurotic (as opposed to the Buddha-mind).)
	\item [Ego inflation] a transitory state in the growth process in which an insight or stage shift frees up mental energy, which makes the student feel free, light, grandiose, bigger than life, seeing things from an exaggerated state of fullness
	\item [Faith] ??? (crossed through? include?)
	\item [Hang-up] (see neurosis)
	\item [Kairos]
	\item [Koan] a brief Zen saying that calls the student to break mental boxes and see things in new way. One of the most common koans is "If you meet the Buddha on the road, kill him."
	\item [LT/Life Training] see MTL
	\item [Life map]
	\item [Love] "a condition of complete simplicity / Costing not less than everything" (T.S. Eliot in \emph{The Four Quartets}); see I Cor. 13; "I define love thus: The will to extend one's self for the purpose of nurturing one's own mind or another's spiritual growth" (p.81, \cite{Peck:78}); cf. "falling in love: a temporary and partial collapse of ego boundaries" (p.90, \cite{Peck:78}); "love is not love which alters when it alteration finds" (Shakespeare)
	\item [MTL/More to Life] a 3 day weekend Kairos workshop, usu.\ in Knoxville or Huntsville, costs \$150-300, student discounts and scholarships available for Dr.\ Hood's Mentor students; a.k.a. Life Training (LT)
	\item [Mindfulness]
	\item [Mindtalk] something the mind tells us that isn�t true (usually the fearful, neurotic voice)
	\item [Minnows]	first-year Mentor students (not a derogatory term!)
	\item [Mysticism] the direct, non-rational experience of what the experiencer calls god (The word is never used in a trivial "new age" sense in the Mentor work. Dr.\ Hood teaches an Honors colloquium on the major mystics of the world's great traditions.)
	\item [Neurosis] a defense against trauma that becomes a habit of mind and behaviors (Carl Rogers)
	\item [Paradox]
	\item [Path]
	\item [Pedestalizing, depedestalizing] In parts of the growth process, the student may put the mentor or others on a pedistal, projecting larger-than-life positions (or, in depedistalizing, very negative) traits on the mentor. Both may be seen as projections of shadow traits in the student, and the pedistal will evaporate as the student moves on in the Work.
	\item [Plus five, minus five (+5,-5)] If we think of consciousness as resting at zero, the more we focus on the outer world, the interaction with it, with noise and excitement, the more the consciousness moves up the scale toward +5. As one withdraws quietly into the self, one drops to -1 or below. The -5 is relative, the deepest one can drop into one's subconscious at a given time.
	\item [Projection] finding faults in others that are true for yourself, despite you may be unable to see them; hypocrically blaming one's own problems on others (Often, giving solutions to others' problems is the advice you need to tell yourself.)
	\item [Religion]		
	\item [Renzai]
	\item [Sangha] Sanskrit, a Buddhist group working together and supporting one another
	\item [Shadow] areas that are denied light; parts of the self that are denied or supressed, often out of fear or misunderstanding (Shadow does not imply a negative connotation in Mentor.)
	\item [Sohbet] Persian, a spiritually bonded group
	\item [Soto]
	\item [Spirituality]
	\item [Transference]
	\item [Validation]
	\item [The Way]	the growth path one follows, esp. the spiritual path (from the Tao)
	\item [The Work] doing the growth work of the individual path(from Gurdjiev)
\end{description}

\section{Quotes}

\section{Works Cited}


%%%%%%%%%%%%%%%%%%%%%%%%%%%%%%%%%%%%%%%%%%%%%%%%%%%%%%%%%%%%%%%%%%%%%%%%%%%%%%%%%%%%%%%%%%%%%%%%%%%%%%%%%%%%%%%%

\section{Further Reading}

%%%%%%%%%%%%%%%%%%%%%%%%%%%%%%%%%%%%%%%%%%%%%%%%%%%%%%%%%%%%%%%%%%%%%%%%%%%%%%%%%%%%%%%%%%%%%%%%%%%%%%%%%%%%%%%%



\bibliographystyle{mla}		% the style you want to use for references.
\bibliography{mr,refs}				% the files containing all the articles and books you ever referenced.
