\chapter{Models of Development (Avery, Luc, Craig, Quinton)}

\begin{authors}
	Avery Edwards, Luc Robinson, \\ \rule{0.5cm}{0cm} Craig Schuff, and Quinton Westrich
\end{authors}

%%%%%%%%%%%%%%%%%%%%%%%%%%%%%%%%%%%%%%%%%%%%%%%%%%%%%%%%%%%%%%%%%%%%%%%%%%%%%%%%%%%%%%%%%%%%%%%%%%%%%%%%%%%%%%%%

\section{\mbox{Motivational Development:} Maslow's Hierarchy of Needs}

%%%%%%%%%%%%%%%%%%%%%%%%%%%%%%%%%%%%%%%%%%%%%%%%%%%%%%%%%%%%%%%%%%%%%%%%%%%%%%%%%%%%%%%%%%%%%%%%%%%%%%%%%%%%%%%%

\subsection{Levels 1-6: Deficiency Needs}

text

\subsubsection{Level 1: Physiological Needs}

text

\subsubsection{Level 2: Safety Needs}

text

\subsubsection{Level 3: Love/Belonging/Social Needs}

text

\subsubsection{Level 4: Esteem Needs}

text

\subsubsection{Level 5: Cognitive Needs}

text

\subsubsection{Level 6: Aesthetic Needs}

text

\subsection{Level 7: Growth Needs}

text

\subsubsection{Level 7: Self-actualization}

text

%%%%%%%%%%%%%%%%%%%%%%%%%%%%%%%%%%%%%%%%%%%%%%%%%%%%%%%%%%%%%%%%%%%%%%%%%%%%%%%%%%%%%%%%%%%%%%%%%%%%%%%%%%%%%%%%

\section{Cognitive and Ethical Development: Perry's Model}

%%%%%%%%%%%%%%%%%%%%%%%%%%%%%%%%%%%%%%%%%%%%%%%%%%%%%%%%%%%%%%%%%%%%%%%%%%%%%%%%%%%%%%%%%%%%%%%%%%%%%%%%%%%%%%%%

\subsection{Context for the Model}
                                             
In 196? William Perry released the first version of his model of cognitive and ethical growth.  With a  sample of Harvard men ages 18 to 25, Perry outlined the progression of these students as they interpreted aspects of their lives in increasingly complex ways.  Perry's model shows "the evolving ways of seeing the world, knowledge, and education, values, and oneself" over the course of the students' college years.  The progression from 1 through 9 is linear, each new stage including and transcending the previous ones.  This model represents a seminal work in the developmental psychology of young adults.  Following Perry's initial findings researchers have further defined and categorized the development of identity and judgment, but Perry's model serves as a baseline for understanding the process as a whole.

\subsection{Stages 1-3: Dualism Modified}

text

\subsubsection{Stage 1: Basic Duality}

A developmental stage where truth and reality is entirely defined by an authority.  Actions and thoughts are divided into right and wrong, two mutually exclusive categories.  Reasoning at this level commonly is demonstrated by parents looking out for the safety of their children.  At young ages crossing the street is wrong as is touching the stove.  The concepts of "dangerous" or "hot" remain too complex and so action is justified or forbidden solely as an arbitrary declaration of the external authority.  

Stage 2 follows quickly after stage 1
"Of course, doesn't everybody?" often marks the introduction of opposing view points backed by other sources of authority. 

\subsubsection{Stage 2: Multiplicity Prelegitimate}

In this stage of reasoning binary categories continue to be the primary way of labeling sets of ideas and actions.  It is inevitable that a human will encounter others who do not think in the same way given the barest of diversity.  However, in Perry 2 anything not a direct product of the chosen authority is wrong, and broader reasoning as seen in later stages is discounted as logically fallacious.  In the face of differing opinions the correct answers are repeated with a citation of the external authority, case closed.  Students in strictly dualist communities may aspire to leadership positions by memorizing the right rules and parroting them back.  If they know and follow the community guide lines well enough then the ascension into the authority position is a sure thing.  Sense of self and identity are often described in relation to the status quo in position and quality. Statements such as "I am a good student," and "I am not as good a student as a should be," are commonplace at this level of reasoning.

"Even the teacher didn't know the answer!"

\subsubsection{Stage 3: Multiplicity Legitimate but Subordinate}

In Perry 3 diversity of thought is seen as unavoidable for the present time.  Even if truth is not clear now, the authorities are working on the problem and will provide the correct answers as they learn them.  In tightly wound Perry 3 hierarchies the exchange of outside opinions and ideas may be labeled as dangerous or corrupting.  Nonmembers may be demonized to discourage interaction or considered thought between the "right" group and the "wrong" group.  This serves to continually polarize groups of people who devalue mutual understanding and compromise.

"What can we really know?"
Following the recognition of present uncertainty is the realization that uncertainty permeates the whole of human knowledge and experience.  Uncertainty is no longer a temporary state but rather an inherent part of being human.  Depending on personality and background Perry 3 moves into 4a or 4b.

\subsection{Stages 4-6: Relativism Discovered}

text

\subsubsection{Stage 4a: Multiplicity (Diversity and Uncertainty) Coordinate}


4. a) Multiplicity (Diversity and Uncertainty) Coordinate with the "known"
In those who come from an authoritarian background the move into Perry four may be accompanied by shock and bitterness towards former authority figures.  It may be seen as a betrayal to have been taught that everything is certain, when it is discovered that everything is not certain or perhaps even nothing is certain.  This distrust may then be extended onto all sources of authority who sound appear to everything figured out.  If all views contain uncertainty, then by what right can anything be taught as correct or true.  As a reconciliation between a new appreciation for diversity and "common sense" rights, a split system  of dualism and multiplicity.  Contained within the dualist right and wrong category are universal human rights as well as those things considered common sense (regardless of actual population percentage that agrees).  Contrary to early dualism, exchange of opinions on proper action and interpretation of these rights is acceptable.  The second category of belief at this stage is all those things shrouded by reasonable uncertainty.  Ideas, beliefs, and ways of life that do not conflict with the human rights of others are given equal value, with no one system held above another.    Valuation of one opinion above another is seen as an arbitrary judgment, and either dismissed or met with indignation.

\subsubsection{Stage 4b: Relativism Subordinate}

4.b)  Relativism Subordinate
For students who have been encouraged to explore uncertainty by their authority figures, Perry 4b is a common next step.  Perry calls the continued trust for authority adherence.   Subordination in this stage  is characterized by deferral to perceived experience and demonstrated wisdom rather than appeal to position.  Even if an authority figure does not know an answer, the relationship may still yield new ways to approach a concept.  This stage marks the beginnings of internalized or personal authority.  Relativism recognizes that some ideas may be more right than others.  Ideas can now be examined within specific context with relevant logic and evidence rather than tackled as universal absolutes.  Ideas are rejected while others are elevated based on individual merit.  Reasonable people may even disagree on particular ideas without personal animosity.  

\subsubsection{Stage 5: Relativism}

\subsubsection{Journal: Craig Schuff, 26 August 2007}

	\begin{quote}"you know, the anger we have for each other is really meaningless. I'm
 tired of carrying this weight around for nothing. Why don't we just drop the
 anger? It's pointless." \\
 Then as a spur of the moment thing, I called up the stairs to their room,
 saying hi, whussup, and would Kent like to come to Cookeville to eat at
 Beethovin's Bistro?
	\end{quote}


You saw the anger in both of you , then saw your part in keeping it alive,
and then did what was needed to change it (name it, claim it, tame it).
Simply Beautiful Scott.

	\begin{quote}
		Yet there is a difference between knowing about it and being there. And
		really you can't know about it without being there.
	\end{quote}

It always turns out somewhat different than we thought, doesn't it?  From
the inside of a place we have a different view as much as we would like to
think we can understand completely from the outside looking in.  On the
first day of Brit Lit Connie likes to say "Be a little humble about what you
think you know."  I have never received better advice than that.

	\begin{quote}
But what was I doing if I wasn't working through Perry 5? I most certainly
 cared about what I was thinking about, not like Perry 5 at all.  Yet I'm not
 saying anything about how "oh, my god. No absolutes to the universe."
	\end{quote}
	
 I think Perry has never really been an all inclusive description of our
work.  It is possible to work on a huge range of concepts from any point on
Perry.  Morality, Divinity, Reality, all of these things.  Perry is a
measure of how much you externalize this process.  So while it doesn't
affect what you think about, it says a little bit about the manner in which
you think and process.  Also caring about how you think is a first step not
a last step, take heart!

Coming into Perry 5 without a context for it or thought for the other side
can make it suck, but you have spent plenty of time playing with a larger
picture.  This can ease the process significantly, taking universal for all
time wtf angst and making it a personal for now feeling.  This is why we
teach Perry!! Huzzah!!


\begin{quote}
 My Perry 5 is when I accepted the meaninglessness and stopped caring,
 period.
\end{quote}

You spent a good deal of time with it all figured out,  a concept of Perry
down to the last detail.  You cared about this picture, you loved it, and
you wanted it to be perfect.  When you let go of that attachment you were
free to just be where you needed to be.  It is freeing and terrifying.  The
future is uncertain when we let ourselves float along down the river but the
alternative is clinging to our stones, in which case we will certainly sink.


We have been rooting for you, Scott.

Fare Forward Traveler,
Craig

\subsubsection{Stage 6: Commitment Forseen}

text

\subsection{Stages 7-9: Commitments in Relativism Developed}

text

\subsubsection{Stage 7: Commitment}

7.Commitment Made (Single)

\subsubsection{Stage 8: Challenges to Commitment}

8.Commitment Cascade

\subsubsection{Stage 9: Evolving Commitment}

9.Relativism Matured

\subsection{Remarks on the Model}

-Core beliefs
-Regressions
	-alienation
	-Retreat to dualism where differences and complexity can be hated
	-Temporizing

\subsection{Alternative Models of Cognitive Development}

\subsubsection{Baxter Magolda's Model}

1992: cognitive development and learning

\subsubsection{King's and Kitchener's Model}

1994: reflective judgment

\subsubsection{Kohlberg's Model}

1969: cognitive and moral development. see p.\pageref{sec:Kohl}

%%%%%%%%%%%%%%%%%%%%%%%%%%%%%%%%%%%%%%%%%%%%%%%%%%%%%%%%%%%%%%%%%%%%%%%%%%%%%%%%%%%%%%%%%%%%%%%%%%%%%%%%%%%%%%%%

\section{Faith and Spiritual Development}

%%%%%%%%%%%%%%%%%%%%%%%%%%%%%%%%%%%%%%%%%%%%%%%%%%%%%%%%%%%%%%%%%%%%%%%%%%%%%%%%%%%%%%%%%%%%%%%%%%%%%%%%%%%%%%%%

\subsection{Fowler's Model of Faith Development}

\subsubsection{Context for the Model}

-James Fowler backgound

      -prof of theology and human development @ Emory in Atlanta

      -great developmental theorists and researchers of our day, work gaining recognition (jacket cover flap) 

\subsubsection{Stage 1: Primal Faith}

infant, birth is trauma, Tillich's "ontological anxiety," 1st symbols of faith: memories of maternal and paternal presence

\subsubsection{Stage 1a: Incorporative Self}

text

\subsubsection{Stage 2: Intuitive-Projective Faith}

about age 2, language emerges, child questions everything, much novelty and newness, free of preformed constructions, pg. 54: "Perception, feelings, and imaginative fantasy make up children's principal ways of knowing---and transforming---their experiences$\ldots$ For now, stimulated by experience and by stories, symbols, and examples, children form deep and longing-lasting images that hold together their worlds of meaning and wonder." , awakening to mystery of death,

\subsubsection{Stage 2a: Impulsive Self}

text

\subsubsection{Stage 3: Mythic Literal Faith}

around 6 or 7 years, relates to development of Piaget's concrete-operational thinking, pg: 55 "stable categories of space, time, and causality make the child's constructions of experience much less dependent on feeling and fantasy.", child becomes more linear, orderly, and predictable, sense of fairness based on reciprocity, faith becomes matter of reliance on stories, rules, implicit values of the family's community of meanings, narrative/story important, people defined by their affiliations and actions, this stage and the rest can typify adults as well

\subsubsection{Stage 3a: Imperial Self}

text

\subsubsection{Stage 4: Synthetic-Conventional Faith}

usually early adolescence, mind takes wings/formal operational thinking, thinking begins to construct ideal possibilities and hypotheticals, imagination, abstract concepts and ideals, systems, able to construc perspectives of others on ourselves, "mutual interpersonal perspective taking"-accounts for "self-consciousness," pulling together of elements, conventional synthesis of elements gotten from one's significant others, one is embedded in his or her faith outlook, which is tacitly held, deeply felt, but mostly unexamined.

\subsubsection{Stage 4a: Interpersonal Self}

text

\subsubsection{Stage 5: Individuative-Reflective Faith}

shift in sense of grounding and orientation of self, emergence of executive ego (behind the personae/masks), objectification and criticial choosing of beliefs/values/commitments

\subsubsection{Stage 5a: Institutional Self}

text

\subsubsection{Stage 6: Conjunctive Faith}

text

\subsubsection{Stage 6a: Inter-Individual Self}

text

\subsubsection{Stage 7: Universalizing Faith}

text

\subsubsection{Stage 7a: God-Grounded Self}

text

\subsection{A Zen Model of Spiritual Development: The Oxherding Pictures}

\subsubsection{Context for the Model}

text

\subsubsection{Stage 1: Undisciplined}

text

\subsubsection{Stage 2: Discipline Begun}

text

\subsubsection{Stage 3: In Harness}

text

\subsubsection{Stage 4: Faced Round}

text

\subsubsection{Stage 5: Tamed}

text

\subsubsection{Stage 6: Unimpeded}

text

\subsubsection{Stage 7: Laissez Faire}

text

\subsubsection{Stage 8: All Forgotten}

text

\subsubsection{Stage 9: The Solitary Moon}

text

\subsubsection{Stage 10: Both Vanished}

text

\section{Connection and Interplay Among the Developmental Models}

text
\section{Quotes}

\section{Works Cited}

%%%%%%%%%%%%%%%%%%%%%%%%%%%%%%%%%%%%%%%%%%%%%%%%%%%%%%%%%%%%%%%%%%%%%%%%%%%%%%%%%%%%%%%%%%%%%%%%%%%%%%%%%%%%%%%%

\section{Further Reading}

%%%%%%%%%%%%%%%%%%%%%%%%%%%%%%%%%%%%%%%%%%%%%%%%%%%%%%%%%%%%%%%%%%%%%%%%%%%%%%%%%%%%%%%%%%%%%%%%%%%%%%%%%%%%%%%%

The original works by Perry are \cite{Per:70} and \cite{Per:81}. \cite{Per:81} gives the revised version of the model which is contained in this chapter. Other models of development include $\ldots$.

The original works by Fowler are \cite{Fow:84} and \cite{Fow:95}.


\bibliographystyle{mla}		% the style you want to use for references.
\bibliography{mr,refs}				% the files containing all the articles and books you ever referenced.
