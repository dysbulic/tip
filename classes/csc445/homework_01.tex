\documentclass[12pt,a4paper,twoside]{article}  % Comments after  % are ignored
\usepackage{amsmath,amssymb,amsfonts}          % Typical maths resource packages
\usepackage{hyperref}                          % For creating hyperlinks in cross references
\usepackage{pstricks}

\pagestyle{headings}         % Option to put page headers
                             % Needed \documentclass[a4paper,twoside]{article}

%\textwidth 17.5cm
%\rightmargin 1in

%\topmargin -1cm
%\parindent 0cm
%\textheight 24cm
%\parskip 1mm

%\newtheorem{definition}[theorem]{Definition}

\author{Will Holcomb \small{CSC445 - Homework \#1}}
\title{Homework \#1}
\date{August 25, 2002}

\begin{document}
\maketitle

\section{Number 1}

Prove that $\overline{(A \cap B)} = \bar A \cup \bar B$

$\overline{(A \cap B)} = \bar A \cup \bar B$ iff
$\overline{(A \cap B)} \subseteq \bar A \cup \bar B$ and
$\overline{(A \cap B)} \supseteq \bar A \cup \bar B$

Let $x \in \overline{(A \cap B)}$, show that $x \in \bar A \cup \bar B$. \\
 Since $x \in \overline{(A \cap B)}$, $x \not \in (A \cap B)$
 by definition of not. \\
 By definition of intersection, $x \not \in A$ or $x \not \in B$. \\
 By definition of union, $x \in \bar A \cup \bar B$. \\
 Since x is arbitrary, by definition of subset,
 $\overline{(A \cap B)} \subseteq \bar A \cup \bar B$.

Let $x \in \bar A \cup \bar B$, show that $x \in \overline{(A \cap B)}$. \\
 Since $x \in \bar A \cup \bar B$, $x \in \bar A$ or $x \in \bar B$
 by definition of union. \\
 By definition of not, $x \not \in A$ or $x \not \in B$  \\
 By definition of intersection, $x \not \in A \cap B$ \\
 By definition of not, $x \in \overline{(A \cap B)}$ \\
 Since x is arbitrary, by definition of subset,
 $\bar A \cup \bar B \subseteq \overline{(A \cap B)}$.

Since $\overline{(A \cap B)} \subseteq \bar A \cup \bar B$ and 
 $\bar A \cup \bar B \subseteq \overline{(A \cap B)}$, 
 $\overline{(A \cap B)} = \bar A \cup \bar B$ by definition of set equality.

\section{Number 2 : Exercise 2.2.1}
\begin{enumerate}

\item The automata can be defined as follows: \\
 $M = (Q, \Sigma, \delta, q_0, F) \ni \\
  Q = \{(x_1, x_2, x_3, a) | x_i \in \{l,r\}; a \in \{A,B\}\} \cup \{q_0\} \\
  \Sigma = \{A,B\}$ \\
 $\delta$ is described by Table~\ref{2.2.1.a} \\
 $F = \hat \delta((r,r,x_3,a), A) \cup \hat \delta((x_1,x_2,r,a), B) \cup
  \hat \delta((x_1,r,l,a), B) \\
  \ni x_i \in \{l,r\}; a \in \{A,B\}$

\begin{table}
\begin{tabular}{c || c | c }
State                             &A           &B         \\
\hline\hline
$>q_0$                            &(r,l,l,A)   &(l,r,r,B) \\
$(l,l,l,x) | x \in \{A,B\}\}$     &(r,l,l,A)   &(l,r,r,B) \\
$(l,l,r,x) | x \in \{A,B\}\}$     &(r,l,r,A)   &(l,r,l,B) \\
$(l,r,l,x) | x \in \{A,B\}\}$     &(r,l,l,A)   &(l,l,r,B) \\
$(l,r,r,x) | x \in \{A,B\}\}$     &(r,r,r,A)   &(l,r,l,B) \\
$(r,l,l,x) | x \in \{A,B\}\}$     &(l,l,l,A)   &(r,r,r,B) \\
$(r,l,r,x) | x \in \{A,B\}\}$     &(l,l,r,A)   &(r,r,l,B) \\
$(r,r,l,x) | x \in \{A,B\}\}$     &(l,r,l,A)   &(r,l,r,B) \\
$(r,r,r,x) | x \in \{A,B\}\}$     &(l,r,r,A)   &(r,r,l,B)
\end{tabular}
\caption{Transitions for $\delta$}\label{2.2.1.a}
\end{table}
\end{enumerate}

\section{Number 3 : Exercise 2.2.4}
\begin{enumerate}

\item The automata can be represented as Figure~\ref{2.2.4.a}
\begin{figure}
% PSTricks TeX macro
% Title: W:\www\csc445\2.2.4.a.dia
% Creator: Dia v0.90
% CreationDate: Wed Aug 28 12:45:01 2002
% For: WJH3957
% \usepackage{pstricks}
% The following commands are not supported in PSTricks at present
% We define them conditionally, so when they are implemented,
% this pstricks file will use them.
\ifx\setlinejoinmode\undefined
  \newcommand{\setlinejoinmode}[1]{}
\fi
\ifx\setlinecaps\undefined
  \newcommand{\setlinecaps}[1]{}
\fi
% This way define your own fonts mapping (for example with ifthen)
\ifx\setfont\undefined
  \newcommand{\setfont}[2]{}
\fi
\pspicture(5.029289,-14.610000)(18.170442,-5.610000)
\scalebox{1.000000 -1.000000}{
\newrgbcolor{dialinecolor}{0.000000 0.000000 0.000000}
\psset{linecolor=dialinecolor}
\newrgbcolor{diafillcolor}{1.000000 1.000000 1.000000}
\psset{fillcolor=diafillcolor}
\psset{linewidth=0.100000}
\psset{linestyle=solid}
\psset{linestyle=solid}
\setlinecaps{0}
\setlinejoinmode{0}
\setlinecaps{0}
\setlinejoinmode{0}
\psset{linestyle=solid}
\newrgbcolor{diafillcolor}{1.000000 1.000000 1.000000}
\psset{fillcolor=diafillcolor}
\pscustom{
\newpath
\moveto(6.103647,9.000000)
\lineto(7.896353,9.000000)
\curveto(8.120442,9.400000)(8.192150,9.600000)(8.192150,10.000000)
\curveto(8.192150,10.400000)(8.120442,10.600000)(7.896353,11.000000)
\lineto(6.103647,11.000000)
\curveto(5.879558,10.600000)(5.807850,10.400000)(5.807850,10.000000)
\curveto(5.807850,9.600000)(5.879558,9.400000)(6.103647,9.000000)
\fill[fillstyle=solid,fillcolor=diafillcolor,linecolor=diafillcolor]}
\newrgbcolor{dialinecolor}{0.000000 0.000000 0.000000}
\psset{linecolor=dialinecolor}
\pscustom{
\newpath
\moveto(6.103647,9.000000)
\lineto(7.896353,9.000000)
\curveto(8.120442,9.400000)(8.192150,9.600000)(8.192150,10.000000)
\curveto(8.192150,10.400000)(8.120442,10.600000)(7.896353,11.000000)
\lineto(6.103647,11.000000)
\curveto(5.879558,10.600000)(5.807850,10.400000)(5.807850,10.000000)
\curveto(5.807850,9.600000)(5.879558,9.400000)(6.103647,9.000000)
\stroke}
\setfont{Courier}{0.800000}
\newrgbcolor{dialinecolor}{0.000000 0.000000 0.000000}
\psset{linecolor=dialinecolor}
\rput(7.000000,10.197980){\scalebox{1 -1}{p0}}
\psset{linewidth=0.100000}
\psset{linestyle=solid}
\psset{linestyle=solid}
\setlinecaps{0}
\setlinejoinmode{0}
\setlinecaps{0}
\setlinejoinmode{0}
\psset{linestyle=solid}
\newrgbcolor{diafillcolor}{1.000000 1.000000 1.000000}
\psset{fillcolor=diafillcolor}
\pscustom{
\newpath
\moveto(11.295797,9.000000)
\lineto(13.088503,9.000000)
\curveto(13.312592,9.400000)(13.384300,9.600000)(13.384300,10.000000)
\curveto(13.384300,10.400000)(13.312592,10.600000)(13.088503,11.000000)
\lineto(11.295797,11.000000)
\curveto(11.071708,10.600000)(11.000000,10.400000)(11.000000,10.000000)
\curveto(11.000000,9.600000)(11.071708,9.400000)(11.295797,9.000000)
\fill[fillstyle=solid,fillcolor=diafillcolor,linecolor=diafillcolor]}
\newrgbcolor{dialinecolor}{0.000000 0.000000 0.000000}
\psset{linecolor=dialinecolor}
\pscustom{
\newpath
\moveto(11.295797,9.000000)
\lineto(13.088503,9.000000)
\curveto(13.312592,9.400000)(13.384300,9.600000)(13.384300,10.000000)
\curveto(13.384300,10.400000)(13.312592,10.600000)(13.088503,11.000000)
\lineto(11.295797,11.000000)
\curveto(11.071708,10.600000)(11.000000,10.400000)(11.000000,10.000000)
\curveto(11.000000,9.600000)(11.071708,9.400000)(11.295797,9.000000)
\stroke}
\setfont{Courier}{0.800000}
\newrgbcolor{dialinecolor}{0.000000 0.000000 0.000000}
\psset{linecolor=dialinecolor}
\rput(12.192150,10.197980){\scalebox{1 -1}{p1}}
\psset{linewidth=0.100000}
\psset{linestyle=solid}
\psset{linestyle=solid}
\setlinecaps{0}
\setlinejoinmode{0}
\setlinecaps{0}
\setlinejoinmode{0}
\psset{linestyle=solid}
\newrgbcolor{dialinecolor}{1.000000 1.000000 1.000000}
\psset{linecolor=dialinecolor}
\pspolygon*(16.148464,9.000000)(17.851536,9.000000)(18.120442,9.300000)(18.120442,10.700000)(17.851536,11.000000)(16.148464,11.000000)(15.879558,10.700000)(15.879558,9.300000)
\newrgbcolor{dialinecolor}{0.000000 0.000000 0.000000}
\psset{linecolor=dialinecolor}
\pspolygon(16.148464,9.000000)(17.851536,9.000000)(18.120442,9.300000)(18.120442,10.700000)(17.851536,11.000000)(16.148464,11.000000)(15.879558,10.700000)(15.879558,9.300000)
\setlinecaps{0}
\setlinejoinmode{0}
\psset{linestyle=solid}
\newrgbcolor{dialinecolor}{1.000000 1.000000 1.000000}
\psset{linecolor=dialinecolor}
\pspolygon*(16.148464,9.150000)(17.851536,9.150000)(17.985989,9.300000)(17.985989,10.700000)(17.851536,10.850000)(16.148464,10.850000)(16.014011,10.700000)(16.014011,9.300000)
\newrgbcolor{dialinecolor}{0.000000 0.000000 0.000000}
\psset{linecolor=dialinecolor}
\pspolygon(16.148464,9.150000)(17.851536,9.150000)(17.985989,9.300000)(17.985989,10.700000)(17.851536,10.850000)(16.148464,10.850000)(16.014011,10.700000)(16.014011,9.300000)
\setfont{Courier}{0.800000}
\newrgbcolor{dialinecolor}{0.000000 0.000000 0.000000}
\psset{linecolor=dialinecolor}
\rput(17.000000,10.197980){\scalebox{1 -1}{p2}}
\psset{linewidth=0.100000}
\psset{linestyle=solid}
\psset{linestyle=solid}
\setlinecaps{0}
\newrgbcolor{dialinecolor}{0.000000 0.000000 0.000000}
\psset{linecolor=dialinecolor}
\psline(8.120442,10.000000)(10.879558,10.000000)
\setlinejoinmode{0}
\newrgbcolor{dialinecolor}{0.000000 0.000000 0.000000}
\psset{linecolor=dialinecolor}
\pspolygon*(10.379558,10.200000)(10.879558,10.000000)(10.379558,9.800000)
\psset{linewidth=0.100000}
\psset{linestyle=solid}
\psset{linestyle=solid}
\setlinecaps{0}
\newrgbcolor{dialinecolor}{0.000000 0.000000 0.000000}
\psset{linecolor=dialinecolor}
\psline(13.312592,10.000000)(15.879558,10.000000)
\setlinejoinmode{0}
\newrgbcolor{dialinecolor}{0.000000 0.000000 0.000000}
\psset{linecolor=dialinecolor}
\pspolygon*(15.379558,10.200000)(15.879558,10.000000)(15.379558,9.800000)
\psset{linewidth=0.100000}
\psset{linestyle=solid}
\psset{linestyle=solid}
\setlinejoinmode{0}
\setlinecaps{0}
\newrgbcolor{dialinecolor}{0.000000 0.000000 0.000000}
\psset{linecolor=dialinecolor}
\pscustom{
\newpath
\moveto(7.000000,9.000000)
\curveto(7.550000,7.300000)(7.050000,6.500000)(6.050000,6.900000)
\curveto(5.050000,7.300000)(7.000000,9.000000)(7.000000,9.000000)
\stroke}
\setlinejoinmode{0}
\newrgbcolor{dialinecolor}{0.000000 0.000000 0.000000}
\psset{linecolor=dialinecolor}
\pspolygon*(6.963621,8.462714)(7.000000,9.000000)(7.344199,8.585842)
\psset{linewidth=0.100000}
\psset{linestyle=solid}
\psset{linestyle=solid}
\setlinecaps{0}
\newrgbcolor{dialinecolor}{0.000000 0.000000 0.000000}
\psset{linecolor=dialinecolor}
\psclip{\pswedge[linestyle=none,fillstyle=none](11.625000,5.704861){10.670438}{44.571175}{135.428825}}
\psellipse(11.625000,5.704861)(7.545139,7.545139)
\endpsclip
\setlinejoinmode{0}
\newrgbcolor{dialinecolor}{0.000000 0.000000 0.000000}
\psset{linecolor=dialinecolor}
\pspolygon*(6.743373,11.215831)(6.250000,11.000000)(6.458422,11.496549)
\psset{linewidth=0.100000}
\psset{linestyle=solid}
\psset{linestyle=solid}
\setlinecaps{0}
\newrgbcolor{dialinecolor}{0.000000 0.000000 0.000000}
\psset{linecolor=dialinecolor}
\psclip{\pswedge[linestyle=none,fillstyle=none](9.596075,7.460526){6.207653}{53.741311}{126.258689}}
\psellipse(9.596075,7.460526)(4.389474,4.389474)
\endpsclip
\setlinejoinmode{0}
\newrgbcolor{dialinecolor}{0.000000 0.000000 0.000000}
\psset{linecolor=dialinecolor}
\pspolygon*(7.521464,11.134445)(7.000000,11.000000)(7.284891,11.456987)
\psset{linewidth=0.100000}
\psset{linestyle=solid}
\psset{linestyle=solid}
\setlinejoinmode{0}
\setlinecaps{0}
\newrgbcolor{dialinecolor}{0.000000 0.000000 0.000000}
\psset{linecolor=dialinecolor}
\pscustom{
\newpath
\moveto(17.000000,9.000000)
\curveto(17.550000,7.300000)(17.145298,6.486434)(16.145298,6.886434)
\curveto(15.145298,7.286434)(17.000000,9.000000)(17.000000,9.000000)
\stroke}
\setlinejoinmode{0}
\newrgbcolor{dialinecolor}{0.000000 0.000000 0.000000}
\psset{linecolor=dialinecolor}
\pspolygon*(16.963621,8.462714)(17.000000,9.000000)(17.344199,8.585842)
\setfont{Times-Roman}{1.000000}
\newrgbcolor{dialinecolor}{0.000000 0.000000 0.000000}
\psset{linecolor=dialinecolor}
\rput(9.450000,9.750000){\scalebox{1 -1}{0}}
\setfont{Times-Roman}{1.000000}
\newrgbcolor{dialinecolor}{0.000000 0.000000 0.000000}
\psset{linecolor=dialinecolor}
\rput(14.296075,9.700000){\scalebox{1 -1}{0}}
\setfont{Times-Roman}{1.000000}
\newrgbcolor{dialinecolor}{0.000000 0.000000 0.000000}
\psset{linecolor=dialinecolor}
\rput(16.750000,6.600000){\scalebox{1 -1}{0}}
\setfont{Times-Roman}{1.000000}
\newrgbcolor{dialinecolor}{0.000000 0.000000 0.000000}
\psset{linecolor=dialinecolor}
\rput(11.800000,14.300000){\scalebox{1 -1}{1}}
\setfont{Times-Roman}{1.000000}
\newrgbcolor{dialinecolor}{0.000000 0.000000 0.000000}
\psset{linecolor=dialinecolor}
\rput(6.350000,6.400000){\scalebox{1 -1}{1}}
\setfont{Times-Roman}{1.000000}
\newrgbcolor{dialinecolor}{0.000000 0.000000 0.000000}
\psset{linecolor=dialinecolor}
\rput(9.652000,11.475000){\scalebox{1 -1}{1}}
\psset{linewidth=0.100000}
\psset{linestyle=solid}
\psset{linestyle=solid}
\setlinejoinmode{0}
\setlinecaps{0}
\newrgbcolor{dialinecolor}{0.000000 0.000000 0.000000}
\psset{linecolor=dialinecolor}
\psline(5.100000,9.400000)(5.700000,10.000000)(5.150000,10.550000)
}\endpspicture
\caption{An automata for all strings ending in 00}\label{2.2.4.a}
\end{figure}

\item The automata can be represented as Figure~\ref{2.2.4.b}
\begin{figure}
% PSTricks TeX macro
% Title: W:\www\csc445\2.2.4.b.dia
% Creator: Dia v0.90
% CreationDate: Wed Aug 28 20:21:07 2002
% For: WJH3957
% \usepackage{pstricks}
% The following commands are not supported in PSTricks at present
% We define them conditionally, so when they are implemented,
% this pstricks file will use them.
\ifx\setlinejoinmode\undefined
  \newcommand{\setlinejoinmode}[1]{}
\fi
\ifx\setlinecaps\undefined
  \newcommand{\setlinecaps}[1]{}
\fi
% This way define your own fonts mapping (for example with ifthen)
\ifx\setfont\undefined
  \newcommand{\setfont}[2]{}
\fi
\pspicture(4.228971,-10.512414)(19.584571,-4.696251)
\scalebox{0.840869 -0.840869}{
\newrgbcolor{dialinecolor}{0.000000 0.000000 0.000000}
\psset{linecolor=dialinecolor}
\newrgbcolor{diafillcolor}{1.000000 1.000000 1.000000}
\psset{fillcolor=diafillcolor}
\psset{linewidth=0.100000}
\psset{linestyle=solid}
\psset{linestyle=solid}
\setlinecaps{0}
\setlinejoinmode{0}
\setlinecaps{0}
\setlinejoinmode{0}
\psset{linestyle=solid}
\newrgbcolor{diafillcolor}{1.000000 1.000000 1.000000}
\psset{fillcolor=diafillcolor}
\pscustom{
\newpath
\moveto(6.103647,9.000000)
\lineto(7.896353,9.000000)
\curveto(8.120442,9.400000)(8.192150,9.600000)(8.192150,10.000000)
\curveto(8.192150,10.400000)(8.120442,10.600000)(7.896353,11.000000)
\lineto(6.103647,11.000000)
\curveto(5.879558,10.600000)(5.807850,10.400000)(5.807850,10.000000)
\curveto(5.807850,9.600000)(5.879558,9.400000)(6.103647,9.000000)
\fill[fillstyle=solid,fillcolor=diafillcolor,linecolor=diafillcolor]}
\newrgbcolor{dialinecolor}{0.000000 0.000000 0.000000}
\psset{linecolor=dialinecolor}
\pscustom{
\newpath
\moveto(6.103647,9.000000)
\lineto(7.896353,9.000000)
\curveto(8.120442,9.400000)(8.192150,9.600000)(8.192150,10.000000)
\curveto(8.192150,10.400000)(8.120442,10.600000)(7.896353,11.000000)
\lineto(6.103647,11.000000)
\curveto(5.879558,10.600000)(5.807850,10.400000)(5.807850,10.000000)
\curveto(5.807850,9.600000)(5.879558,9.400000)(6.103647,9.000000)
\stroke}
\setfont{Courier}{0.800000}
\newrgbcolor{dialinecolor}{0.000000 0.000000 0.000000}
\psset{linecolor=dialinecolor}
\rput(7.000000,10.197980){\scalebox{1 -1}{p0}}
\psset{linewidth=0.100000}
\psset{linestyle=solid}
\psset{linestyle=solid}
\setlinecaps{0}
\setlinejoinmode{0}
\setlinecaps{0}
\setlinejoinmode{0}
\psset{linestyle=solid}
\newrgbcolor{diafillcolor}{1.000000 1.000000 1.000000}
\psset{fillcolor=diafillcolor}
\pscustom{
\newpath
\moveto(11.295797,9.000000)
\lineto(13.088503,9.000000)
\curveto(13.312592,9.400000)(13.384300,9.600000)(13.384300,10.000000)
\curveto(13.384300,10.400000)(13.312592,10.600000)(13.088503,11.000000)
\lineto(11.295797,11.000000)
\curveto(11.071708,10.600000)(11.000000,10.400000)(11.000000,10.000000)
\curveto(11.000000,9.600000)(11.071708,9.400000)(11.295797,9.000000)
\fill[fillstyle=solid,fillcolor=diafillcolor,linecolor=diafillcolor]}
\newrgbcolor{dialinecolor}{0.000000 0.000000 0.000000}
\psset{linecolor=dialinecolor}
\pscustom{
\newpath
\moveto(11.295797,9.000000)
\lineto(13.088503,9.000000)
\curveto(13.312592,9.400000)(13.384300,9.600000)(13.384300,10.000000)
\curveto(13.384300,10.400000)(13.312592,10.600000)(13.088503,11.000000)
\lineto(11.295797,11.000000)
\curveto(11.071708,10.600000)(11.000000,10.400000)(11.000000,10.000000)
\curveto(11.000000,9.600000)(11.071708,9.400000)(11.295797,9.000000)
\stroke}
\setfont{Courier}{0.800000}
\newrgbcolor{dialinecolor}{0.000000 0.000000 0.000000}
\psset{linecolor=dialinecolor}
\rput(12.192150,10.197980){\scalebox{1 -1}{p1}}
\psset{linewidth=0.100000}
\psset{linestyle=solid}
\psset{linestyle=solid}
\setlinecaps{0}
\setlinejoinmode{0}
\setlinecaps{0}
\setlinejoinmode{0}
\psset{linestyle=solid}
\newrgbcolor{dialinecolor}{1.000000 1.000000 1.000000}
\psset{linecolor=dialinecolor}
\pspolygon*(21.268905,9.000000)(22.971975,9.000000)(23.240881,9.300000)(23.240881,10.700000)(22.971975,11.000000)(21.268905,11.000000)(20.999999,10.700000)(20.999999,9.300000)
\newrgbcolor{dialinecolor}{0.000000 0.000000 0.000000}
\psset{linecolor=dialinecolor}
\pspolygon(21.268905,9.000000)(22.971975,9.000000)(23.240881,9.300000)(23.240881,10.700000)(22.971975,11.000000)(21.268905,11.000000)(20.999999,10.700000)(20.999999,9.300000)
\setlinecaps{0}
\setlinejoinmode{0}
\psset{linestyle=solid}
\newrgbcolor{dialinecolor}{1.000000 1.000000 1.000000}
\psset{linecolor=dialinecolor}
\pspolygon*(21.268905,9.150000)(22.971975,9.150000)(23.106428,9.300000)(23.106428,10.700000)(22.971975,10.850000)(21.268905,10.850000)(21.134452,10.700000)(21.134452,9.300000)
\newrgbcolor{dialinecolor}{0.000000 0.000000 0.000000}
\psset{linecolor=dialinecolor}
\pspolygon(21.268905,9.150000)(22.971975,9.150000)(23.106428,9.300000)(23.106428,10.700000)(22.971975,10.850000)(21.268905,10.850000)(21.134452,10.700000)(21.134452,9.300000)
\setfont{Courier}{0.800000}
\newrgbcolor{dialinecolor}{0.000000 0.000000 0.000000}
\psset{linecolor=dialinecolor}
\rput(22.120440,10.197980){\scalebox{1 -1}{p3}}
\psset{linewidth=0.100000}
\psset{linestyle=solid}
\psset{linestyle=solid}
\setlinecaps{0}
\newrgbcolor{dialinecolor}{0.000000 0.000000 0.000000}
\psset{linecolor=dialinecolor}
\psline(8.120440,10.000000)(10.879600,10.000000)
\setlinejoinmode{0}
\newrgbcolor{dialinecolor}{0.000000 0.000000 0.000000}
\psset{linecolor=dialinecolor}
\pspolygon*(10.379600,10.200000)(10.879600,10.000000)(10.379600,9.800000)
\psset{linewidth=0.100000}
\psset{linestyle=solid}
\psset{linestyle=solid}
\setlinecaps{0}
\newrgbcolor{dialinecolor}{0.000000 0.000000 0.000000}
\psset{linecolor=dialinecolor}
\psline(13.312600,10.000000)(16.000000,10.000000)
\setlinejoinmode{0}
\newrgbcolor{dialinecolor}{0.000000 0.000000 0.000000}
\psset{linecolor=dialinecolor}
\pspolygon*(15.500000,10.200000)(16.000000,10.000000)(15.500000,9.800000)
\psset{linewidth=0.100000}
\psset{linestyle=solid}
\psset{linestyle=solid}
\setlinejoinmode{0}
\setlinecaps{0}
\newrgbcolor{dialinecolor}{0.000000 0.000000 0.000000}
\psset{linecolor=dialinecolor}
\pscustom{
\newpath
\moveto(7.000000,9.000000)
\curveto(7.550000,7.300000)(7.050000,6.500000)(6.050000,6.900000)
\curveto(5.050000,7.300000)(7.000000,9.000000)(7.000000,9.000000)
\stroke}
\setlinejoinmode{0}
\newrgbcolor{dialinecolor}{0.000000 0.000000 0.000000}
\psset{linecolor=dialinecolor}
\pspolygon*(6.963621,8.462714)(7.000000,9.000000)(7.344199,8.585842)
\psset{linewidth=0.100000}
\psset{linestyle=solid}
\psset{linestyle=solid}
\setlinecaps{0}
\newrgbcolor{dialinecolor}{0.000000 0.000000 0.000000}
\psset{linecolor=dialinecolor}
\psclip{\pswedge[linestyle=none,fillstyle=none](10.096100,8.840509){4.256063}{45.853415}{134.146585}}
\psellipse(10.096100,8.840509)(3.009491,3.009491)
\endpsclip
\setlinejoinmode{0}
\newrgbcolor{dialinecolor}{0.000000 0.000000 0.000000}
\psset{linecolor=dialinecolor}
\pspolygon*(8.498079,11.204736)(8.000000,11.000000)(8.219481,11.491760)
\psset{linewidth=0.100000}
\psset{linestyle=solid}
\psset{linestyle=solid}
\setlinejoinmode{0}
\setlinecaps{0}
\newrgbcolor{dialinecolor}{0.000000 0.000000 0.000000}
\psset{linecolor=dialinecolor}
\pscustom{
\newpath
\moveto(22.120400,9.000000)
\curveto(22.670400,7.300000)(22.345300,6.386430)(21.345300,6.786430)
\curveto(20.345300,7.186430)(22.120400,9.000000)(22.120400,9.000000)
\stroke}
\setlinejoinmode{0}
\newrgbcolor{dialinecolor}{0.000000 0.000000 0.000000}
\psset{linecolor=dialinecolor}
\pspolygon*(22.084021,8.462714)(22.120400,9.000000)(22.464599,8.585842)
\setfont{Times-Roman}{1.000000}
\newrgbcolor{dialinecolor}{0.000000 0.000000 0.000000}
\psset{linecolor=dialinecolor}
\rput(9.450000,9.750000){\scalebox{1 -1}{0}}
\setfont{Times-Roman}{1.000000}
\newrgbcolor{dialinecolor}{0.000000 0.000000 0.000000}
\psset{linecolor=dialinecolor}
\rput(14.296100,9.700000){\scalebox{1 -1}{0}}
\setfont{Times-Roman}{1.000000}
\newrgbcolor{dialinecolor}{0.000000 0.000000 0.000000}
\psset{linecolor=dialinecolor}
\rput(19.400000,9.750000){\scalebox{1 -1}{0}}
\setfont{Times-Roman}{1.000000}
\newrgbcolor{dialinecolor}{0.000000 0.000000 0.000000}
\psset{linecolor=dialinecolor}
\rput(6.350000,6.400000){\scalebox{1 -1}{1}}
\setfont{Times-Roman}{1.000000}
\newrgbcolor{dialinecolor}{0.000000 0.000000 0.000000}
\psset{linecolor=dialinecolor}
\rput(9.652000,11.475000){\scalebox{1 -1}{1}}
\psset{linewidth=0.100000}
\psset{linestyle=solid}
\psset{linestyle=solid}
\setlinejoinmode{0}
\setlinecaps{0}
\newrgbcolor{dialinecolor}{0.000000 0.000000 0.000000}
\psset{linecolor=dialinecolor}
\psline(5.100000,9.400000)(5.700000,10.000000)(5.150000,10.550000)
\psset{linewidth=0.100000}
\psset{linestyle=solid}
\psset{linestyle=solid}
\setlinecaps{0}
\newrgbcolor{dialinecolor}{0.000000 0.000000 0.000000}
\psset{linecolor=dialinecolor}
\psline(18.312600,10.000000)(21.000000,10.000000)
\setlinejoinmode{0}
\newrgbcolor{dialinecolor}{0.000000 0.000000 0.000000}
\psset{linecolor=dialinecolor}
\pspolygon*(20.500000,10.200000)(21.000000,10.000000)(20.500000,9.800000)
\psset{linewidth=0.100000}
\psset{linestyle=solid}
\psset{linestyle=solid}
\setlinecaps{0}
\setlinejoinmode{0}
\setlinecaps{0}
\setlinejoinmode{0}
\psset{linestyle=solid}
\newrgbcolor{diafillcolor}{1.000000 1.000000 1.000000}
\psset{fillcolor=diafillcolor}
\pscustom{
\newpath
\moveto(16.295797,9.000000)
\lineto(18.088503,9.000000)
\curveto(18.312592,9.400000)(18.384300,9.600000)(18.384300,10.000000)
\curveto(18.384300,10.400000)(18.312592,10.600000)(18.088503,11.000000)
\lineto(16.295797,11.000000)
\curveto(16.071708,10.600000)(16.000000,10.400000)(16.000000,10.000000)
\curveto(16.000000,9.600000)(16.071708,9.400000)(16.295797,9.000000)
\fill[fillstyle=solid,fillcolor=diafillcolor,linecolor=diafillcolor]}
\newrgbcolor{dialinecolor}{0.000000 0.000000 0.000000}
\psset{linecolor=dialinecolor}
\pscustom{
\newpath
\moveto(16.295797,9.000000)
\lineto(18.088503,9.000000)
\curveto(18.312592,9.400000)(18.384300,9.600000)(18.384300,10.000000)
\curveto(18.384300,10.400000)(18.312592,10.600000)(18.088503,11.000000)
\lineto(16.295797,11.000000)
\curveto(16.071708,10.600000)(16.000000,10.400000)(16.000000,10.000000)
\curveto(16.000000,9.600000)(16.071708,9.400000)(16.295797,9.000000)
\stroke}
\setfont{Courier}{0.800000}
\newrgbcolor{dialinecolor}{0.000000 0.000000 0.000000}
\psset{linecolor=dialinecolor}
\rput(17.192150,10.197980){\scalebox{1 -1}{p2}}
\psset{linewidth=0.100000}
\psset{linestyle=solid}
\psset{linestyle=solid}
\setlinecaps{0}
\newrgbcolor{dialinecolor}{0.000000 0.000000 0.000000}
\psset{linecolor=dialinecolor}
\psclip{\pswedge[linestyle=none,fillstyle=none](12.155897,2.990748){13.380019}{57.837883}{121.588717}}
\psellipse(12.155897,2.990748)(9.461102,9.461102)
\endpsclip
\setlinejoinmode{0}
\newrgbcolor{dialinecolor}{0.000000 0.000000 0.000000}
\psset{linecolor=dialinecolor}
\pspolygon*(7.730679,11.141543)(7.200000,11.050000)(7.521151,11.482275)
\setfont{Times-Roman}{1.000000}
\newrgbcolor{dialinecolor}{0.000000 0.000000 0.000000}
\psset{linecolor=dialinecolor}
\rput(12.402000,12.175000){\scalebox{1 -1}{1}}
\setfont{Times-Roman}{1.000000}
\newrgbcolor{dialinecolor}{0.000000 0.000000 0.000000}
\psset{linecolor=dialinecolor}
\rput(21.702000,6.375000){\scalebox{1 -1}{0,1}}
}\endpspicture
\caption{An automata for all strings with 000 as a substring}\label{2.2.4.b}
\end{figure}

\item The automama can be represented as  Figure~\ref{2.2.4.c}
\begin{figure}
\input{2.2.4.c.tex}
\caption{An automata for all strings with 011 as a substring}\label{2.2.4.c}
\end{figure}
\end{enumerate}

\section{Number 4 : Exercise 2.2.5}
\begin{enumerate}
\item The automata can be defined as follows: [This is for each block
 of 5 containing at least two zeroes. $(|w| \% 5 = 0)$] \\
 $M = (Q, \Sigma, \delta, q_0, F) \ni \\
  Q = \{(z, c) | z,c \in \mathbb{N}; z,c < 5\} \\
  \Sigma = \{0, 1\} \\
  \delta((z, c), 0) = (z + 1, c + 1) \ni c \leq 3 \\
  \delta((z, c), 1) = (z, c + 1) \ni c \leq 3 \\
  \delta((z, 4), 0) = (0, 0) \ni z \geq 1 \\
  \delta((z, 4), 1) = (0, 0) \ni z \geq 2 \\
  q_0 = (0, 0) \\
  F = \{(0, 0)\}$

%[For any block of five containing two zeroes:] \\
% $\delta((z, c), 0) = (z + 1, c + 1) \ni z = 0; c \leq 3 \\
%  \delta((z, c), 1) = (z, c + 1) \ni c \leq 3 \\
%  \delta((z, c), 0) = (0, 0) \ni z = 1; c \leq 4$ \\

\item The automata can be represented as Figure~\ref{2.2.5.b}
\begin{figure}
% PSTricks TeX macro
% Title: W:\www\csc445\2.2.5.b.dia
% Creator: Dia v0.90
% CreationDate: Wed Aug 28 20:22:48 2002
% For: WJH3957
% \usepackage{pstricks}
% The following commands are not supported in PSTricks at present
% We define them conditionally, so when they are implemented,
% this pstricks file will use them.
\ifx\setlinejoinmode\undefined
  \newcommand{\setlinejoinmode}[1]{}
\fi
\ifx\setlinecaps\undefined
  \newcommand{\setlinecaps}[1]{}
\fi
% This way define your own fonts mapping (for example with ifthen)
\ifx\setfont\undefined
  \newcommand{\setfont}[2]{}
\fi
\pspicture(2.818016,-9.553471)(18.173615,-3.143400)
\scalebox{0.560321 -0.560321}{
\newrgbcolor{dialinecolor}{0.000000 0.000000 0.000000}
\psset{linecolor=dialinecolor}
\newrgbcolor{diafillcolor}{1.000000 1.000000 1.000000}
\psset{fillcolor=diafillcolor}
\psset{linewidth=0.100000}
\psset{linestyle=solid}
\psset{linestyle=solid}
\setlinecaps{0}
\setlinejoinmode{0}
\setlinecaps{0}
\setlinejoinmode{0}
\psset{linestyle=solid}
\newrgbcolor{diafillcolor}{1.000000 1.000000 1.000000}
\psset{fillcolor=diafillcolor}
\pscustom{
\newpath
\moveto(6.103647,9.000000)
\lineto(7.896353,9.000000)
\curveto(8.120442,9.400000)(8.192150,9.600000)(8.192150,10.000000)
\curveto(8.192150,10.400000)(8.120442,10.600000)(7.896353,11.000000)
\lineto(6.103647,11.000000)
\curveto(5.879558,10.600000)(5.807850,10.400000)(5.807850,10.000000)
\curveto(5.807850,9.600000)(5.879558,9.400000)(6.103647,9.000000)
\fill[fillstyle=solid,fillcolor=diafillcolor,linecolor=diafillcolor]}
\newrgbcolor{dialinecolor}{0.000000 0.000000 0.000000}
\psset{linecolor=dialinecolor}
\pscustom{
\newpath
\moveto(6.103647,9.000000)
\lineto(7.896353,9.000000)
\curveto(8.120442,9.400000)(8.192150,9.600000)(8.192150,10.000000)
\curveto(8.192150,10.400000)(8.120442,10.600000)(7.896353,11.000000)
\lineto(6.103647,11.000000)
\curveto(5.879558,10.600000)(5.807850,10.400000)(5.807850,10.000000)
\curveto(5.807850,9.600000)(5.879558,9.400000)(6.103647,9.000000)
\stroke}
\setfont{Courier}{0.800000}
\newrgbcolor{dialinecolor}{0.000000 0.000000 0.000000}
\psset{linecolor=dialinecolor}
\rput(7.000000,10.197980){\scalebox{1 -1}{p0}}
\psset{linewidth=0.100000}
\psset{linestyle=solid}
\psset{linestyle=solid}
\setlinecaps{0}
\setlinejoinmode{0}
\setlinecaps{0}
\setlinejoinmode{0}
\psset{linestyle=solid}
\newrgbcolor{diafillcolor}{1.000000 1.000000 1.000000}
\psset{fillcolor=diafillcolor}
\pscustom{
\newpath
\moveto(10.295797,9.000000)
\lineto(12.088503,9.000000)
\curveto(12.312592,9.400000)(12.384300,9.600000)(12.384300,10.000000)
\curveto(12.384300,10.400000)(12.312592,10.600000)(12.088503,11.000000)
\lineto(10.295797,11.000000)
\curveto(10.071708,10.600000)(10.000000,10.400000)(10.000000,10.000000)
\curveto(10.000000,9.600000)(10.071708,9.400000)(10.295797,9.000000)
\fill[fillstyle=solid,fillcolor=diafillcolor,linecolor=diafillcolor]}
\newrgbcolor{dialinecolor}{0.000000 0.000000 0.000000}
\psset{linecolor=dialinecolor}
\pscustom{
\newpath
\moveto(10.295797,9.000000)
\lineto(12.088503,9.000000)
\curveto(12.312592,9.400000)(12.384300,9.600000)(12.384300,10.000000)
\curveto(12.384300,10.400000)(12.312592,10.600000)(12.088503,11.000000)
\lineto(10.295797,11.000000)
\curveto(10.071708,10.600000)(10.000000,10.400000)(10.000000,10.000000)
\curveto(10.000000,9.600000)(10.071708,9.400000)(10.295797,9.000000)
\stroke}
\setfont{Courier}{0.800000}
\newrgbcolor{dialinecolor}{0.000000 0.000000 0.000000}
\psset{linecolor=dialinecolor}
\rput(11.192150,10.197980){\scalebox{1 -1}{p1}}
\psset{linewidth=0.100000}
\psset{linestyle=solid}
\psset{linestyle=solid}
\setlinecaps{0}
\setlinejoinmode{0}
\setlinecaps{0}
\setlinejoinmode{0}
\psset{linestyle=solid}
\newrgbcolor{dialinecolor}{1.000000 1.000000 1.000000}
\psset{linecolor=dialinecolor}
\pspolygon*(20.416746,15.000000)(23.056134,15.000000)(23.472880,15.300000)(23.472880,16.700000)(23.056134,17.000000)(20.416746,17.000000)(20.000000,16.700000)(20.000000,15.300000)
\newrgbcolor{dialinecolor}{0.000000 0.000000 0.000000}
\psset{linecolor=dialinecolor}
\pspolygon(20.416746,15.000000)(23.056134,15.000000)(23.472880,15.300000)(23.472880,16.700000)(23.056134,17.000000)(20.416746,17.000000)(20.000000,16.700000)(20.000000,15.300000)
\setlinecaps{0}
\setlinejoinmode{0}
\psset{linestyle=solid}
\newrgbcolor{dialinecolor}{1.000000 1.000000 1.000000}
\psset{linecolor=dialinecolor}
\pspolygon*(20.416746,15.150000)(23.056134,15.150000)(23.264507,15.300000)(23.264507,16.700000)(23.056134,16.850000)(20.416746,16.850000)(20.208373,16.700000)(20.208373,15.300000)
\newrgbcolor{dialinecolor}{0.000000 0.000000 0.000000}
\psset{linecolor=dialinecolor}
\pspolygon(20.416746,15.150000)(23.056134,15.150000)(23.264507,15.300000)(23.264507,16.700000)(23.056134,16.850000)(20.416746,16.850000)(20.208373,16.700000)(20.208373,15.300000)
\setfont{Courier}{0.800000}
\newrgbcolor{dialinecolor}{0.000000 0.000000 0.000000}
\psset{linecolor=dialinecolor}
\rput(21.736440,16.197980){\scalebox{1 -1}{p10}}
\psset{linewidth=0.100000}
\psset{linestyle=solid}
\psset{linestyle=solid}
\setlinecaps{0}
\newrgbcolor{dialinecolor}{0.000000 0.000000 0.000000}
\psset{linecolor=dialinecolor}
\psline(8.120440,10.000000)(10.000000,10.000000)
\setlinejoinmode{0}
\newrgbcolor{dialinecolor}{0.000000 0.000000 0.000000}
\psset{linecolor=dialinecolor}
\pspolygon*(9.500000,10.200000)(10.000000,10.000000)(9.500000,9.800000)
\psset{linewidth=0.100000}
\psset{linestyle=solid}
\psset{linestyle=solid}
\setlinecaps{0}
\newrgbcolor{dialinecolor}{0.000000 0.000000 0.000000}
\psset{linecolor=dialinecolor}
\psline(12.312600,10.000000)(14.000000,10.000000)
\setlinejoinmode{0}
\newrgbcolor{dialinecolor}{0.000000 0.000000 0.000000}
\psset{linecolor=dialinecolor}
\pspolygon*(13.500000,10.200000)(14.000000,10.000000)(13.500000,9.800000)
\psset{linewidth=0.100000}
\psset{linestyle=solid}
\psset{linestyle=solid}
\setlinejoinmode{0}
\setlinecaps{0}
\newrgbcolor{dialinecolor}{0.000000 0.000000 0.000000}
\psset{linecolor=dialinecolor}
\pscustom{
\newpath
\moveto(7.000000,9.000000)
\curveto(7.550000,7.300000)(7.050000,6.500000)(6.050000,6.900000)
\curveto(5.050000,7.300000)(7.000000,9.000000)(7.000000,9.000000)
\stroke}
\setlinejoinmode{0}
\newrgbcolor{dialinecolor}{0.000000 0.000000 0.000000}
\psset{linecolor=dialinecolor}
\pspolygon*(6.963621,8.462714)(7.000000,9.000000)(7.344199,8.585842)
\psset{linewidth=0.100000}
\psset{linestyle=solid}
\psset{linestyle=solid}
\setlinejoinmode{0}
\setlinecaps{0}
\newrgbcolor{dialinecolor}{0.000000 0.000000 0.000000}
\psset{linecolor=dialinecolor}
\pscustom{
\newpath
\moveto(9.192150,15.000000)
\curveto(9.229290,13.369500)(7.679290,13.519500)(19.179300,13.469500)
\curveto(30.679300,13.419500)(31.192100,11.000000)(31.192100,11.000000)
\stroke}
\setlinejoinmode{0}
\newrgbcolor{dialinecolor}{0.000000 0.000000 0.000000}
\psset{linecolor=dialinecolor}
\pspolygon*(9.003588,14.495575)(9.192150,15.000000)(9.403484,14.504684)
\setfont{Times-Roman}{1.000000}
\newrgbcolor{dialinecolor}{0.000000 0.000000 0.000000}
\psset{linecolor=dialinecolor}
\rput(8.950000,9.584760){\scalebox{1 -1}{1}}
\setfont{Times-Roman}{1.000000}
\newrgbcolor{dialinecolor}{0.000000 0.000000 0.000000}
\psset{linecolor=dialinecolor}
\rput(13.006300,9.584760){\scalebox{1 -1}{0,1}}
\setfont{Times-Roman}{1.000000}
\newrgbcolor{dialinecolor}{0.000000 0.000000 0.000000}
\psset{linecolor=dialinecolor}
\rput(6.350000,6.400000){\scalebox{1 -1}{0,1}}
\psset{linewidth=0.100000}
\psset{linestyle=solid}
\psset{linestyle=solid}
\setlinejoinmode{0}
\setlinecaps{0}
\newrgbcolor{dialinecolor}{0.000000 0.000000 0.000000}
\psset{linecolor=dialinecolor}
\psline(5.100000,9.400000)(5.700000,10.000000)(5.150000,10.550000)
\psset{linewidth=0.100000}
\psset{linestyle=solid}
\psset{linestyle=solid}
\setlinecaps{0}
\setlinejoinmode{0}
\setlinecaps{0}
\setlinejoinmode{0}
\psset{linestyle=solid}
\newrgbcolor{diafillcolor}{1.000000 1.000000 1.000000}
\psset{fillcolor=diafillcolor}
\pscustom{
\newpath
\moveto(14.295797,9.000000)
\lineto(16.088503,9.000000)
\curveto(16.312592,9.400000)(16.384300,9.600000)(16.384300,10.000000)
\curveto(16.384300,10.400000)(16.312592,10.600000)(16.088503,11.000000)
\lineto(14.295797,11.000000)
\curveto(14.071708,10.600000)(14.000000,10.400000)(14.000000,10.000000)
\curveto(14.000000,9.600000)(14.071708,9.400000)(14.295797,9.000000)
\fill[fillstyle=solid,fillcolor=diafillcolor,linecolor=diafillcolor]}
\newrgbcolor{dialinecolor}{0.000000 0.000000 0.000000}
\psset{linecolor=dialinecolor}
\pscustom{
\newpath
\moveto(14.295797,9.000000)
\lineto(16.088503,9.000000)
\curveto(16.312592,9.400000)(16.384300,9.600000)(16.384300,10.000000)
\curveto(16.384300,10.400000)(16.312592,10.600000)(16.088503,11.000000)
\lineto(14.295797,11.000000)
\curveto(14.071708,10.600000)(14.000000,10.400000)(14.000000,10.000000)
\curveto(14.000000,9.600000)(14.071708,9.400000)(14.295797,9.000000)
\stroke}
\setfont{Courier}{0.800000}
\newrgbcolor{dialinecolor}{0.000000 0.000000 0.000000}
\psset{linecolor=dialinecolor}
\rput(15.192150,10.197980){\scalebox{1 -1}{p2}}
\psset{linewidth=0.100000}
\psset{linestyle=solid}
\psset{linestyle=solid}
\setlinecaps{0}
\newrgbcolor{dialinecolor}{0.000000 0.000000 0.000000}
\psset{linecolor=dialinecolor}
\psline(16.312600,10.000000)(18.000000,10.000000)
\setlinejoinmode{0}
\newrgbcolor{dialinecolor}{0.000000 0.000000 0.000000}
\psset{linecolor=dialinecolor}
\pspolygon*(17.500000,10.200000)(18.000000,10.000000)(17.500000,9.800000)
\setfont{Times-Roman}{1.000000}
\newrgbcolor{dialinecolor}{0.000000 0.000000 0.000000}
\psset{linecolor=dialinecolor}
\rput(17.006300,9.584760){\scalebox{1 -1}{0,1}}
\psset{linewidth=0.100000}
\psset{linestyle=solid}
\psset{linestyle=solid}
\setlinecaps{0}
\setlinejoinmode{0}
\setlinecaps{0}
\setlinejoinmode{0}
\psset{linestyle=solid}
\newrgbcolor{diafillcolor}{1.000000 1.000000 1.000000}
\psset{fillcolor=diafillcolor}
\pscustom{
\newpath
\moveto(18.295797,9.000000)
\lineto(20.088503,9.000000)
\curveto(20.312592,9.400000)(20.384300,9.600000)(20.384300,10.000000)
\curveto(20.384300,10.400000)(20.312592,10.600000)(20.088503,11.000000)
\lineto(18.295797,11.000000)
\curveto(18.071708,10.600000)(18.000000,10.400000)(18.000000,10.000000)
\curveto(18.000000,9.600000)(18.071708,9.400000)(18.295797,9.000000)
\fill[fillstyle=solid,fillcolor=diafillcolor,linecolor=diafillcolor]}
\newrgbcolor{dialinecolor}{0.000000 0.000000 0.000000}
\psset{linecolor=dialinecolor}
\pscustom{
\newpath
\moveto(18.295797,9.000000)
\lineto(20.088503,9.000000)
\curveto(20.312592,9.400000)(20.384300,9.600000)(20.384300,10.000000)
\curveto(20.384300,10.400000)(20.312592,10.600000)(20.088503,11.000000)
\lineto(18.295797,11.000000)
\curveto(18.071708,10.600000)(18.000000,10.400000)(18.000000,10.000000)
\curveto(18.000000,9.600000)(18.071708,9.400000)(18.295797,9.000000)
\stroke}
\setfont{Courier}{0.800000}
\newrgbcolor{dialinecolor}{0.000000 0.000000 0.000000}
\psset{linecolor=dialinecolor}
\rput(19.192150,10.197980){\scalebox{1 -1}{p3}}
\psset{linewidth=0.100000}
\psset{linestyle=solid}
\psset{linestyle=solid}
\setlinecaps{0}
\newrgbcolor{dialinecolor}{0.000000 0.000000 0.000000}
\psset{linecolor=dialinecolor}
\psline(20.312600,10.000000)(22.000000,10.000000)
\setlinejoinmode{0}
\newrgbcolor{dialinecolor}{0.000000 0.000000 0.000000}
\psset{linecolor=dialinecolor}
\pspolygon*(21.500000,10.200000)(22.000000,10.000000)(21.500000,9.800000)
\setfont{Times-Roman}{1.000000}
\newrgbcolor{dialinecolor}{0.000000 0.000000 0.000000}
\psset{linecolor=dialinecolor}
\rput(21.006300,9.584760){\scalebox{1 -1}{0,1}}
\psset{linewidth=0.100000}
\psset{linestyle=solid}
\psset{linestyle=solid}
\setlinecaps{0}
\setlinejoinmode{0}
\setlinecaps{0}
\setlinejoinmode{0}
\psset{linestyle=solid}
\newrgbcolor{diafillcolor}{1.000000 1.000000 1.000000}
\psset{fillcolor=diafillcolor}
\pscustom{
\newpath
\moveto(22.295797,9.000000)
\lineto(24.088503,9.000000)
\curveto(24.312592,9.400000)(24.384300,9.600000)(24.384300,10.000000)
\curveto(24.384300,10.400000)(24.312592,10.600000)(24.088503,11.000000)
\lineto(22.295797,11.000000)
\curveto(22.071708,10.600000)(22.000000,10.400000)(22.000000,10.000000)
\curveto(22.000000,9.600000)(22.071708,9.400000)(22.295797,9.000000)
\fill[fillstyle=solid,fillcolor=diafillcolor,linecolor=diafillcolor]}
\newrgbcolor{dialinecolor}{0.000000 0.000000 0.000000}
\psset{linecolor=dialinecolor}
\pscustom{
\newpath
\moveto(22.295797,9.000000)
\lineto(24.088503,9.000000)
\curveto(24.312592,9.400000)(24.384300,9.600000)(24.384300,10.000000)
\curveto(24.384300,10.400000)(24.312592,10.600000)(24.088503,11.000000)
\lineto(22.295797,11.000000)
\curveto(22.071708,10.600000)(22.000000,10.400000)(22.000000,10.000000)
\curveto(22.000000,9.600000)(22.071708,9.400000)(22.295797,9.000000)
\stroke}
\setfont{Courier}{0.800000}
\newrgbcolor{dialinecolor}{0.000000 0.000000 0.000000}
\psset{linecolor=dialinecolor}
\rput(23.192150,10.197980){\scalebox{1 -1}{p4}}
\psset{linewidth=0.100000}
\psset{linestyle=solid}
\psset{linestyle=solid}
\setlinecaps{0}
\newrgbcolor{dialinecolor}{0.000000 0.000000 0.000000}
\psset{linecolor=dialinecolor}
\psline(24.312600,10.000000)(26.000000,10.000000)
\setlinejoinmode{0}
\newrgbcolor{dialinecolor}{0.000000 0.000000 0.000000}
\psset{linecolor=dialinecolor}
\pspolygon*(25.500000,10.200000)(26.000000,10.000000)(25.500000,9.800000)
\setfont{Times-Roman}{1.000000}
\newrgbcolor{dialinecolor}{0.000000 0.000000 0.000000}
\psset{linecolor=dialinecolor}
\rput(25.029300,9.584760){\scalebox{1 -1}{0,1}}
\psset{linewidth=0.100000}
\psset{linestyle=solid}
\psset{linestyle=solid}
\setlinecaps{0}
\setlinejoinmode{0}
\setlinecaps{0}
\setlinejoinmode{0}
\psset{linestyle=solid}
\newrgbcolor{diafillcolor}{1.000000 1.000000 1.000000}
\psset{fillcolor=diafillcolor}
\pscustom{
\newpath
\moveto(26.295797,9.000000)
\lineto(28.088503,9.000000)
\curveto(28.312592,9.400000)(28.384300,9.600000)(28.384300,10.000000)
\curveto(28.384300,10.400000)(28.312592,10.600000)(28.088503,11.000000)
\lineto(26.295797,11.000000)
\curveto(26.071708,10.600000)(26.000000,10.400000)(26.000000,10.000000)
\curveto(26.000000,9.600000)(26.071708,9.400000)(26.295797,9.000000)
\fill[fillstyle=solid,fillcolor=diafillcolor,linecolor=diafillcolor]}
\newrgbcolor{dialinecolor}{0.000000 0.000000 0.000000}
\psset{linecolor=dialinecolor}
\pscustom{
\newpath
\moveto(26.295797,9.000000)
\lineto(28.088503,9.000000)
\curveto(28.312592,9.400000)(28.384300,9.600000)(28.384300,10.000000)
\curveto(28.384300,10.400000)(28.312592,10.600000)(28.088503,11.000000)
\lineto(26.295797,11.000000)
\curveto(26.071708,10.600000)(26.000000,10.400000)(26.000000,10.000000)
\curveto(26.000000,9.600000)(26.071708,9.400000)(26.295797,9.000000)
\stroke}
\setfont{Courier}{0.800000}
\newrgbcolor{dialinecolor}{0.000000 0.000000 0.000000}
\psset{linecolor=dialinecolor}
\rput(27.192150,10.197980){\scalebox{1 -1}{p5}}
\psset{linewidth=0.100000}
\psset{linestyle=solid}
\psset{linestyle=solid}
\setlinecaps{0}
\newrgbcolor{dialinecolor}{0.000000 0.000000 0.000000}
\psset{linecolor=dialinecolor}
\psline(28.312600,10.000000)(30.000000,10.000000)
\setlinejoinmode{0}
\newrgbcolor{dialinecolor}{0.000000 0.000000 0.000000}
\psset{linecolor=dialinecolor}
\pspolygon*(29.500000,10.200000)(30.000000,10.000000)(29.500000,9.800000)
\setfont{Times-Roman}{1.000000}
\newrgbcolor{dialinecolor}{0.000000 0.000000 0.000000}
\psset{linecolor=dialinecolor}
\rput(29.006300,9.584760){\scalebox{1 -1}{0,1}}
\psset{linewidth=0.100000}
\psset{linestyle=solid}
\psset{linestyle=solid}
\setlinecaps{0}
\setlinejoinmode{0}
\setlinecaps{0}
\setlinejoinmode{0}
\psset{linestyle=solid}
\newrgbcolor{diafillcolor}{1.000000 1.000000 1.000000}
\psset{fillcolor=diafillcolor}
\pscustom{
\newpath
\moveto(30.295797,9.000000)
\lineto(32.088503,9.000000)
\curveto(32.312592,9.400000)(32.384300,9.600000)(32.384300,10.000000)
\curveto(32.384300,10.400000)(32.312592,10.600000)(32.088503,11.000000)
\lineto(30.295797,11.000000)
\curveto(30.071708,10.600000)(30.000000,10.400000)(30.000000,10.000000)
\curveto(30.000000,9.600000)(30.071708,9.400000)(30.295797,9.000000)
\fill[fillstyle=solid,fillcolor=diafillcolor,linecolor=diafillcolor]}
\newrgbcolor{dialinecolor}{0.000000 0.000000 0.000000}
\psset{linecolor=dialinecolor}
\pscustom{
\newpath
\moveto(30.295797,9.000000)
\lineto(32.088503,9.000000)
\curveto(32.312592,9.400000)(32.384300,9.600000)(32.384300,10.000000)
\curveto(32.384300,10.400000)(32.312592,10.600000)(32.088503,11.000000)
\lineto(30.295797,11.000000)
\curveto(30.071708,10.600000)(30.000000,10.400000)(30.000000,10.000000)
\curveto(30.000000,9.600000)(30.071708,9.400000)(30.295797,9.000000)
\stroke}
\setfont{Courier}{0.800000}
\newrgbcolor{dialinecolor}{0.000000 0.000000 0.000000}
\psset{linecolor=dialinecolor}
\rput(31.192150,10.197980){\scalebox{1 -1}{p6}}
\setfont{Times-Roman}{1.000000}
\newrgbcolor{dialinecolor}{0.000000 0.000000 0.000000}
\psset{linecolor=dialinecolor}
\rput(17.479300,13.019500){\scalebox{1 -1}{0,1}}
\psset{linewidth=0.100000}
\psset{linestyle=solid}
\psset{linestyle=solid}
\setlinecaps{0}
\setlinejoinmode{0}
\setlinecaps{0}
\setlinejoinmode{0}
\psset{linestyle=solid}
\newrgbcolor{diafillcolor}{1.000000 1.000000 1.000000}
\psset{fillcolor=diafillcolor}
\pscustom{
\newpath
\moveto(8.295797,15.000000)
\lineto(10.088503,15.000000)
\curveto(10.312592,15.400000)(10.384300,15.600000)(10.384300,16.000000)
\curveto(10.384300,16.400000)(10.312592,16.600000)(10.088503,17.000000)
\lineto(8.295797,17.000000)
\curveto(8.071708,16.600000)(8.000000,16.400000)(8.000000,16.000000)
\curveto(8.000000,15.600000)(8.071708,15.400000)(8.295797,15.000000)
\fill[fillstyle=solid,fillcolor=diafillcolor,linecolor=diafillcolor]}
\newrgbcolor{dialinecolor}{0.000000 0.000000 0.000000}
\psset{linecolor=dialinecolor}
\pscustom{
\newpath
\moveto(8.295797,15.000000)
\lineto(10.088503,15.000000)
\curveto(10.312592,15.400000)(10.384300,15.600000)(10.384300,16.000000)
\curveto(10.384300,16.400000)(10.312592,16.600000)(10.088503,17.000000)
\lineto(8.295797,17.000000)
\curveto(8.071708,16.600000)(8.000000,16.400000)(8.000000,16.000000)
\curveto(8.000000,15.600000)(8.071708,15.400000)(8.295797,15.000000)
\stroke}
\setfont{Courier}{0.800000}
\newrgbcolor{dialinecolor}{0.000000 0.000000 0.000000}
\psset{linecolor=dialinecolor}
\rput(9.192150,16.197980){\scalebox{1 -1}{p7}}
\psset{linewidth=0.100000}
\psset{linestyle=solid}
\psset{linestyle=solid}
\setlinecaps{0}
\newrgbcolor{dialinecolor}{0.000000 0.000000 0.000000}
\psset{linecolor=dialinecolor}
\psline(10.312600,16.000000)(12.000000,16.000000)
\setlinejoinmode{0}
\newrgbcolor{dialinecolor}{0.000000 0.000000 0.000000}
\psset{linecolor=dialinecolor}
\pspolygon*(11.500000,16.200000)(12.000000,16.000000)(11.500000,15.800000)
\setfont{Times-Roman}{1.000000}
\newrgbcolor{dialinecolor}{0.000000 0.000000 0.000000}
\psset{linecolor=dialinecolor}
\rput(11.056300,15.650000){\scalebox{1 -1}{0,1}}
\psset{linewidth=0.100000}
\psset{linestyle=solid}
\psset{linestyle=solid}
\setlinecaps{0}
\setlinejoinmode{0}
\setlinecaps{0}
\setlinejoinmode{0}
\psset{linestyle=solid}
\newrgbcolor{diafillcolor}{1.000000 1.000000 1.000000}
\psset{fillcolor=diafillcolor}
\pscustom{
\newpath
\moveto(12.295797,15.000000)
\lineto(14.088503,15.000000)
\curveto(14.312592,15.400000)(14.384300,15.600000)(14.384300,16.000000)
\curveto(14.384300,16.400000)(14.312592,16.600000)(14.088503,17.000000)
\lineto(12.295797,17.000000)
\curveto(12.071708,16.600000)(12.000000,16.400000)(12.000000,16.000000)
\curveto(12.000000,15.600000)(12.071708,15.400000)(12.295797,15.000000)
\fill[fillstyle=solid,fillcolor=diafillcolor,linecolor=diafillcolor]}
\newrgbcolor{dialinecolor}{0.000000 0.000000 0.000000}
\psset{linecolor=dialinecolor}
\pscustom{
\newpath
\moveto(12.295797,15.000000)
\lineto(14.088503,15.000000)
\curveto(14.312592,15.400000)(14.384300,15.600000)(14.384300,16.000000)
\curveto(14.384300,16.400000)(14.312592,16.600000)(14.088503,17.000000)
\lineto(12.295797,17.000000)
\curveto(12.071708,16.600000)(12.000000,16.400000)(12.000000,16.000000)
\curveto(12.000000,15.600000)(12.071708,15.400000)(12.295797,15.000000)
\stroke}
\setfont{Courier}{0.800000}
\newrgbcolor{dialinecolor}{0.000000 0.000000 0.000000}
\psset{linecolor=dialinecolor}
\rput(13.192150,16.197980){\scalebox{1 -1}{p8}}
\psset{linewidth=0.100000}
\psset{linestyle=solid}
\psset{linestyle=solid}
\setlinecaps{0}
\newrgbcolor{dialinecolor}{0.000000 0.000000 0.000000}
\psset{linecolor=dialinecolor}
\psline(14.312600,16.000000)(16.000000,16.000000)
\setlinejoinmode{0}
\newrgbcolor{dialinecolor}{0.000000 0.000000 0.000000}
\psset{linecolor=dialinecolor}
\pspolygon*(15.500000,16.200000)(16.000000,16.000000)(15.500000,15.800000)
\setfont{Times-Roman}{1.000000}
\newrgbcolor{dialinecolor}{0.000000 0.000000 0.000000}
\psset{linecolor=dialinecolor}
\rput(15.156300,15.675000){\scalebox{1 -1}{0,1}}
\psset{linewidth=0.100000}
\psset{linestyle=solid}
\psset{linestyle=solid}
\setlinecaps{0}
\setlinejoinmode{0}
\setlinecaps{0}
\setlinejoinmode{0}
\psset{linestyle=solid}
\newrgbcolor{diafillcolor}{1.000000 1.000000 1.000000}
\psset{fillcolor=diafillcolor}
\pscustom{
\newpath
\moveto(16.295797,15.000000)
\lineto(18.088503,15.000000)
\curveto(18.312592,15.400000)(18.384300,15.600000)(18.384300,16.000000)
\curveto(18.384300,16.400000)(18.312592,16.600000)(18.088503,17.000000)
\lineto(16.295797,17.000000)
\curveto(16.071708,16.600000)(16.000000,16.400000)(16.000000,16.000000)
\curveto(16.000000,15.600000)(16.071708,15.400000)(16.295797,15.000000)
\fill[fillstyle=solid,fillcolor=diafillcolor,linecolor=diafillcolor]}
\newrgbcolor{dialinecolor}{0.000000 0.000000 0.000000}
\psset{linecolor=dialinecolor}
\pscustom{
\newpath
\moveto(16.295797,15.000000)
\lineto(18.088503,15.000000)
\curveto(18.312592,15.400000)(18.384300,15.600000)(18.384300,16.000000)
\curveto(18.384300,16.400000)(18.312592,16.600000)(18.088503,17.000000)
\lineto(16.295797,17.000000)
\curveto(16.071708,16.600000)(16.000000,16.400000)(16.000000,16.000000)
\curveto(16.000000,15.600000)(16.071708,15.400000)(16.295797,15.000000)
\stroke}
\setfont{Courier}{0.800000}
\newrgbcolor{dialinecolor}{0.000000 0.000000 0.000000}
\psset{linecolor=dialinecolor}
\rput(17.192150,16.197980){\scalebox{1 -1}{p9}}
\psset{linewidth=0.100000}
\psset{linestyle=solid}
\psset{linestyle=solid}
\setlinecaps{0}
\newrgbcolor{dialinecolor}{0.000000 0.000000 0.000000}
\psset{linecolor=dialinecolor}
\psline(18.312600,16.000000)(20.000000,16.000000)
\setlinejoinmode{0}
\newrgbcolor{dialinecolor}{0.000000 0.000000 0.000000}
\psset{linecolor=dialinecolor}
\pspolygon*(19.500000,16.200000)(20.000000,16.000000)(19.500000,15.800000)
\setfont{Times-Roman}{1.000000}
\newrgbcolor{dialinecolor}{0.000000 0.000000 0.000000}
\psset{linecolor=dialinecolor}
\rput(19.206300,15.675000){\scalebox{1 -1}{0,1}}
}\endpspicture
\caption{An automata for strings with 1 as the 10th character to the right}\label{2.2.5.b}
\end{figure}

\item The automata can be represented as Figure~\ref{2.2.5.c}
\begin{figure}
% PSTricks TeX macro
% Title: X:\www\csc445\2.2.5.c.dia
% Creator: Dia v0.90
% CreationDate: Thu Aug 29 17:22:44 2002
% For: student
% \usepackage{pstricks}
% The following commands are not supported in PSTricks at present
% We define them conditionally, so when they are implemented,
% this pstricks file will use them.
\ifx\setlinejoinmode\undefined
  \newcommand{\setlinejoinmode}[1]{}
\fi
\ifx\setlinecaps\undefined
  \newcommand{\setlinecaps}[1]{}
\fi
% This way define your own fonts mapping (for example with ifthen)
\ifx\setfont\undefined
  \newcommand{\setfont}[2]{}
\fi
\pspicture(4.958217,-16.520540)(20.313817,-8.458277)
\scalebox{0.985868 -0.985868}{
\newrgbcolor{dialinecolor}{0.000000 0.000000 0.000000}
\psset{linecolor=dialinecolor}
\newrgbcolor{diafillcolor}{1.000000 1.000000 1.000000}
\psset{fillcolor=diafillcolor}
\psset{linewidth=0.100000}
\psset{linestyle=solid}
\psset{linestyle=solid}
\setlinecaps{0}
\setlinejoinmode{0}
\setlinecaps{0}
\setlinejoinmode{0}
\psset{linestyle=solid}
\newrgbcolor{diafillcolor}{1.000000 1.000000 1.000000}
\psset{fillcolor=diafillcolor}
\pscustom{
\newpath
\moveto(6.103647,9.000000)
\lineto(7.896353,9.000000)
\curveto(8.120442,9.400000)(8.192150,9.600000)(8.192150,10.000000)
\curveto(8.192150,10.400000)(8.120442,10.600000)(7.896353,11.000000)
\lineto(6.103647,11.000000)
\curveto(5.879558,10.600000)(5.807850,10.400000)(5.807850,10.000000)
\curveto(5.807850,9.600000)(5.879558,9.400000)(6.103647,9.000000)
\fill[fillstyle=solid,fillcolor=diafillcolor,linecolor=diafillcolor]}
\newrgbcolor{dialinecolor}{0.000000 0.000000 0.000000}
\psset{linecolor=dialinecolor}
\pscustom{
\newpath
\moveto(6.103647,9.000000)
\lineto(7.896353,9.000000)
\curveto(8.120442,9.400000)(8.192150,9.600000)(8.192150,10.000000)
\curveto(8.192150,10.400000)(8.120442,10.600000)(7.896353,11.000000)
\lineto(6.103647,11.000000)
\curveto(5.879558,10.600000)(5.807850,10.400000)(5.807850,10.000000)
\curveto(5.807850,9.600000)(5.879558,9.400000)(6.103647,9.000000)
\stroke}
\setfont{Courier}{0.800000}
\newrgbcolor{dialinecolor}{0.000000 0.000000 0.000000}
\psset{linecolor=dialinecolor}
\rput(7.000000,10.197980){\scalebox{1 -1}{p0}}
\psset{linewidth=0.100000}
\psset{linestyle=solid}
\psset{linestyle=solid}
\setlinecaps{0}
\setlinejoinmode{0}
\setlinecaps{0}
\setlinejoinmode{0}
\psset{linestyle=solid}
\newrgbcolor{diafillcolor}{1.000000 1.000000 1.000000}
\psset{fillcolor=diafillcolor}
\pscustom{
\newpath
\moveto(10.295797,9.000000)
\lineto(12.088503,9.000000)
\curveto(12.312592,9.400000)(12.384300,9.600000)(12.384300,10.000000)
\curveto(12.384300,10.400000)(12.312592,10.600000)(12.088503,11.000000)
\lineto(10.295797,11.000000)
\curveto(10.071708,10.600000)(10.000000,10.400000)(10.000000,10.000000)
\curveto(10.000000,9.600000)(10.071708,9.400000)(10.295797,9.000000)
\fill[fillstyle=solid,fillcolor=diafillcolor,linecolor=diafillcolor]}
\newrgbcolor{dialinecolor}{0.000000 0.000000 0.000000}
\psset{linecolor=dialinecolor}
\pscustom{
\newpath
\moveto(10.295797,9.000000)
\lineto(12.088503,9.000000)
\curveto(12.312592,9.400000)(12.384300,9.600000)(12.384300,10.000000)
\curveto(12.384300,10.400000)(12.312592,10.600000)(12.088503,11.000000)
\lineto(10.295797,11.000000)
\curveto(10.071708,10.600000)(10.000000,10.400000)(10.000000,10.000000)
\curveto(10.000000,9.600000)(10.071708,9.400000)(10.295797,9.000000)
\stroke}
\setfont{Courier}{0.800000}
\newrgbcolor{dialinecolor}{0.000000 0.000000 0.000000}
\psset{linecolor=dialinecolor}
\rput(11.192150,10.197980){\scalebox{1 -1}{p1}}
\psset{linewidth=0.100000}
\psset{linestyle=solid}
\psset{linestyle=solid}
\setlinecaps{0}
\newrgbcolor{dialinecolor}{0.000000 0.000000 0.000000}
\psset{linecolor=dialinecolor}
\psline(8.120442,10.000000)(10.000000,10.000000)
\setlinejoinmode{0}
\newrgbcolor{dialinecolor}{0.000000 0.000000 0.000000}
\psset{linecolor=dialinecolor}
\pspolygon*(9.500000,10.200000)(10.000000,10.000000)(9.500000,9.800000)
\psset{linewidth=0.100000}
\psset{linestyle=solid}
\psset{linestyle=solid}
\setlinecaps{0}
\newrgbcolor{dialinecolor}{0.000000 0.000000 0.000000}
\psset{linecolor=dialinecolor}
\psline(12.312592,10.000000)(13.999999,10.000000)
\setlinejoinmode{0}
\newrgbcolor{dialinecolor}{0.000000 0.000000 0.000000}
\psset{linecolor=dialinecolor}
\pspolygon*(13.499999,10.200000)(13.999999,10.000000)(13.499999,9.800000)
\setfont{Times-Roman}{1.000000}
\newrgbcolor{dialinecolor}{0.000000 0.000000 0.000000}
\psset{linecolor=dialinecolor}
\rput(9.029290,9.419520){\scalebox{1 -1}{0}}
\setfont{Times-Roman}{1.000000}
\newrgbcolor{dialinecolor}{0.000000 0.000000 0.000000}
\psset{linecolor=dialinecolor}
\rput(13.006300,9.400000){\scalebox{1 -1}{1}}
\psset{linewidth=0.100000}
\psset{linestyle=solid}
\psset{linestyle=solid}
\setlinejoinmode{0}
\setlinecaps{0}
\newrgbcolor{dialinecolor}{0.000000 0.000000 0.000000}
\psset{linecolor=dialinecolor}
\psline(5.100000,9.400000)(5.700000,10.000000)(5.150000,10.550000)
\setfont{Times-Roman}{1.000000}
\newrgbcolor{dialinecolor}{0.000000 0.000000 0.000000}
\psset{linecolor=dialinecolor}
\rput(20.000000,10.000000){\scalebox{1 -1}{0,1}}
\psset{linewidth=0.100000}
\psset{linestyle=solid}
\psset{linestyle=solid}
\setlinecaps{0}
\setlinejoinmode{0}
\setlinecaps{0}
\setlinejoinmode{0}
\psset{linestyle=solid}
\newrgbcolor{dialinecolor}{1.000000 1.000000 1.000000}
\psset{linecolor=dialinecolor}
\pspolygon*(14.268905,9.000000)(15.971975,9.000000)(16.240881,9.300000)(16.240881,10.700000)(15.971975,11.000000)(14.268905,11.000000)(13.999999,10.700000)(13.999999,9.300000)
\newrgbcolor{dialinecolor}{0.000000 0.000000 0.000000}
\psset{linecolor=dialinecolor}
\pspolygon(14.268905,9.000000)(15.971975,9.000000)(16.240881,9.300000)(16.240881,10.700000)(15.971975,11.000000)(14.268905,11.000000)(13.999999,10.700000)(13.999999,9.300000)
\setlinecaps{0}
\setlinejoinmode{0}
\psset{linestyle=solid}
\newrgbcolor{dialinecolor}{1.000000 1.000000 1.000000}
\psset{linecolor=dialinecolor}
\pspolygon*(14.268905,9.150000)(15.971975,9.150000)(16.106428,9.300000)(16.106428,10.700000)(15.971975,10.850000)(14.268905,10.850000)(14.134452,10.700000)(14.134452,9.300000)
\newrgbcolor{dialinecolor}{0.000000 0.000000 0.000000}
\psset{linecolor=dialinecolor}
\pspolygon(14.268905,9.150000)(15.971975,9.150000)(16.106428,9.300000)(16.106428,10.700000)(15.971975,10.850000)(14.268905,10.850000)(14.134452,10.700000)(14.134452,9.300000)
\setfont{Courier}{0.800000}
\newrgbcolor{dialinecolor}{0.000000 0.000000 0.000000}
\psset{linecolor=dialinecolor}
\rput(15.120440,10.197980){\scalebox{1 -1}{p2}}
\psset{linewidth=0.100000}
\psset{linestyle=solid}
\psset{linestyle=solid}
\setlinejoinmode{0}
\setlinecaps{0}
\newrgbcolor{dialinecolor}{0.000000 0.000000 0.000000}
\psset{linecolor=dialinecolor}
\pscustom{
\newpath
\moveto(16.000000,10.000000)
\curveto(18.538400,8.604520)(18.800000,9.050000)(19.000000,10.000000)
\curveto(19.200000,10.950000)(16.240881,10.000000)(16.240881,10.000000)
\stroke}
\setlinejoinmode{0}
\newrgbcolor{dialinecolor}{0.000000 0.000000 0.000000}
\psset{linecolor=dialinecolor}
\pspolygon*(16.341805,9.583864)(16.000000,10.000000)(16.534504,9.934387)
\psset{linewidth=0.100000}
\psset{linestyle=solid}
\psset{linestyle=solid}
\setlinecaps{0}
\newrgbcolor{dialinecolor}{0.000000 0.000000 0.000000}
\psset{linecolor=dialinecolor}
\psline(7.000000,11.000000)(9.000000,13.000000)
\setlinejoinmode{0}
\newrgbcolor{dialinecolor}{0.000000 0.000000 0.000000}
\psset{linecolor=dialinecolor}
\pspolygon*(8.505025,12.787868)(9.000000,13.000000)(8.787868,12.505025)
\psset{linewidth=0.100000}
\psset{linestyle=solid}
\psset{linestyle=solid}
\setlinecaps{0}
\setlinejoinmode{0}
\setlinecaps{0}
\setlinejoinmode{0}
\psset{linestyle=solid}
\newrgbcolor{diafillcolor}{1.000000 1.000000 1.000000}
\psset{fillcolor=diafillcolor}
\pscustom{
\newpath
\moveto(9.295797,12.000000)
\lineto(11.088503,12.000000)
\curveto(11.312592,12.400000)(11.384300,12.600000)(11.384300,13.000000)
\curveto(11.384300,13.400000)(11.312592,13.600000)(11.088503,14.000000)
\lineto(9.295797,14.000000)
\curveto(9.071708,13.600000)(9.000000,13.400000)(9.000000,13.000000)
\curveto(9.000000,12.600000)(9.071708,12.400000)(9.295797,12.000000)
\fill[fillstyle=solid,fillcolor=diafillcolor,linecolor=diafillcolor]}
\newrgbcolor{dialinecolor}{0.000000 0.000000 0.000000}
\psset{linecolor=dialinecolor}
\pscustom{
\newpath
\moveto(9.295797,12.000000)
\lineto(11.088503,12.000000)
\curveto(11.312592,12.400000)(11.384300,12.600000)(11.384300,13.000000)
\curveto(11.384300,13.400000)(11.312592,13.600000)(11.088503,14.000000)
\lineto(9.295797,14.000000)
\curveto(9.071708,13.600000)(9.000000,13.400000)(9.000000,13.000000)
\curveto(9.000000,12.600000)(9.071708,12.400000)(9.295797,12.000000)
\stroke}
\setfont{Courier}{0.800000}
\newrgbcolor{dialinecolor}{0.000000 0.000000 0.000000}
\psset{linecolor=dialinecolor}
\rput(10.192150,13.197980){\scalebox{1 -1}{p3}}
\psset{linewidth=0.100000}
\psset{linestyle=solid}
\psset{linestyle=solid}
\setlinecaps{0}
\newrgbcolor{dialinecolor}{0.000000 0.000000 0.000000}
\psset{linecolor=dialinecolor}
\psline(11.312592,13.000000)(13.000000,13.000000)
\setlinejoinmode{0}
\newrgbcolor{dialinecolor}{0.000000 0.000000 0.000000}
\psset{linecolor=dialinecolor}
\pspolygon*(12.500000,13.200000)(13.000000,13.000000)(12.500000,12.800000)
\setfont{Times-Roman}{1.000000}
\newrgbcolor{dialinecolor}{0.000000 0.000000 0.000000}
\psset{linecolor=dialinecolor}
\rput(7.350000,12.700000){\scalebox{1 -1}{0,1}}
\psset{linewidth=0.100000}
\psset{linestyle=solid}
\psset{linestyle=solid}
\setlinecaps{0}
\setlinejoinmode{0}
\setlinecaps{0}
\setlinejoinmode{0}
\psset{linestyle=solid}
\newrgbcolor{diafillcolor}{1.000000 1.000000 1.000000}
\psset{fillcolor=diafillcolor}
\pscustom{
\newpath
\moveto(13.295797,12.000000)
\lineto(15.088503,12.000000)
\curveto(15.312592,12.400000)(15.384300,12.600000)(15.384300,13.000000)
\curveto(15.384300,13.400000)(15.312592,13.600000)(15.088503,14.000000)
\lineto(13.295797,14.000000)
\curveto(13.071708,13.600000)(13.000000,13.400000)(13.000000,13.000000)
\curveto(13.000000,12.600000)(13.071708,12.400000)(13.295797,12.000000)
\fill[fillstyle=solid,fillcolor=diafillcolor,linecolor=diafillcolor]}
\newrgbcolor{dialinecolor}{0.000000 0.000000 0.000000}
\psset{linecolor=dialinecolor}
\pscustom{
\newpath
\moveto(13.295797,12.000000)
\lineto(15.088503,12.000000)
\curveto(15.312592,12.400000)(15.384300,12.600000)(15.384300,13.000000)
\curveto(15.384300,13.400000)(15.312592,13.600000)(15.088503,14.000000)
\lineto(13.295797,14.000000)
\curveto(13.071708,13.600000)(13.000000,13.400000)(13.000000,13.000000)
\curveto(13.000000,12.600000)(13.071708,12.400000)(13.295797,12.000000)
\stroke}
\setfont{Courier}{0.800000}
\newrgbcolor{dialinecolor}{0.000000 0.000000 0.000000}
\psset{linecolor=dialinecolor}
\rput(14.192150,13.197980){\scalebox{1 -1}{p4}}
\setfont{Times-Roman}{1.000000}
\newrgbcolor{dialinecolor}{0.000000 0.000000 0.000000}
\psset{linecolor=dialinecolor}
\rput(12.000000,14.000000){\scalebox{1 -1}{0}}
\setfont{Times-Roman}{1.000000}
\newrgbcolor{dialinecolor}{0.000000 0.000000 0.000000}
\psset{linecolor=dialinecolor}
\rput(11.965804,16.447350){\scalebox{1 -1}{0,1}}
\psset{linewidth=0.100000}
\psset{linestyle=solid}
\psset{linestyle=solid}
\setlinejoinmode{0}
\setlinecaps{0}
\newrgbcolor{dialinecolor}{0.000000 0.000000 0.000000}
\psset{linecolor=dialinecolor}
\pscustom{
\newpath
\moveto(10.479289,14.060000)
\curveto(12.079289,16.010000)(10.729303,16.460000)(10.229303,16.260000)
\curveto(9.729303,16.060000)(10.192150,14.000000)(10.192150,14.000000)
\stroke}
\setlinejoinmode{0}
\newrgbcolor{dialinecolor}{0.000000 0.000000 0.000000}
\psset{linecolor=dialinecolor}
\pspolygon*(10.951062,14.319673)(10.479289,14.060000)(10.641833,14.573400)
\psset{linewidth=0.100000}
\psset{linestyle=solid}
\psset{linestyle=solid}
\setlinecaps{0}
\newrgbcolor{dialinecolor}{0.000000 0.000000 0.000000}
\psset{linecolor=dialinecolor}
\psline(15.312592,13.000000)(17.000000,13.000000)
\setlinejoinmode{0}
\newrgbcolor{dialinecolor}{0.000000 0.000000 0.000000}
\psset{linecolor=dialinecolor}
\pspolygon*(16.500000,13.200000)(17.000000,13.000000)(16.500000,12.800000)
\psset{linewidth=0.100000}
\psset{linestyle=solid}
\psset{linestyle=solid}
\setlinecaps{0}
\setlinejoinmode{0}
\setlinecaps{0}
\setlinejoinmode{0}
\psset{linestyle=solid}
\newrgbcolor{dialinecolor}{1.000000 1.000000 1.000000}
\psset{linecolor=dialinecolor}
\pspolygon*(17.268906,12.000000)(18.971977,12.000000)(19.240883,12.300000)(19.240883,13.700000)(18.971977,14.000000)(17.268906,14.000000)(17.000000,13.700000)(17.000000,12.300000)
\newrgbcolor{dialinecolor}{0.000000 0.000000 0.000000}
\psset{linecolor=dialinecolor}
\pspolygon(17.268906,12.000000)(18.971977,12.000000)(19.240883,12.300000)(19.240883,13.700000)(18.971977,14.000000)(17.268906,14.000000)(17.000000,13.700000)(17.000000,12.300000)
\setlinecaps{0}
\setlinejoinmode{0}
\psset{linestyle=solid}
\newrgbcolor{dialinecolor}{1.000000 1.000000 1.000000}
\psset{linecolor=dialinecolor}
\pspolygon*(17.268906,12.150000)(18.971977,12.150000)(19.106430,12.300000)(19.106430,13.700000)(18.971977,13.850000)(17.268906,13.850000)(17.134453,13.700000)(17.134453,12.300000)
\newrgbcolor{dialinecolor}{0.000000 0.000000 0.000000}
\psset{linecolor=dialinecolor}
\pspolygon(17.268906,12.150000)(18.971977,12.150000)(19.106430,12.300000)(19.106430,13.700000)(18.971977,13.850000)(17.268906,13.850000)(17.134453,13.700000)(17.134453,12.300000)
\setfont{Courier}{0.800000}
\newrgbcolor{dialinecolor}{0.000000 0.000000 0.000000}
\psset{linecolor=dialinecolor}
\rput(18.120441,13.197980){\scalebox{1 -1}{p2}}
\setfont{Times-Roman}{1.000000}
\newrgbcolor{dialinecolor}{0.000000 0.000000 0.000000}
\psset{linecolor=dialinecolor}
\rput(16.000000,14.000000){\scalebox{1 -1}{1}}
}\endpspicture
\caption{An automata for strings that begin and/or end with 01}\label{2.2.5.c}
\end{figure}

\item The automata can be defined as follows: \\
 $M = (Q, \Sigma, \delta, q_0, F) \ni \\
  Q = \{(z, n) | z,n \in \mathbb{N}; z < 5; n < 3\} \\
  \Sigma = \{0,1\} \\
  \delta((z,n), 0) = ((z + 1) \% 5, n) \\
  \delta((z,n), 1) = (z, (n + 1) \% 3) \\
  q_0 = (0,0) \\
  F = \{(0,0)\}$

\end{enumerate}

\section{Number 5 : Exercise 2.3.5}

Prove that assuming if $\delta_D(q, a) = p$ then $\delta_N(q, a) =
 \{p\}$, it follows that if $\hat \delta_D(q_0, w) = p$ then $\hat
 \delta_N(q_0,w) = \{p\}$. \\
 Start with $w = \epsilon$. By definition, $\hat \delta_D(q_0,
 \epsilon) = q_0$ and $\hat \delta_N(q_0, \epsilon) = \{q_0\}$.  \\
 Consider then $|w| \geq 1$; let $x$ be a string and $a$ a letter such
 that $w = xa$. \\
 By the inductive hypothesis and the definition of $\hat \delta_D$,
 $\hat \delta_D(q_0, w) = \hat \delta_D(q_0, xa) = \delta_D(\hat
 \delta(q_0, x), a)$. \\
 By the inductive hypothesis and the definition of $\hat \delta_N$,
 $\hat \delta_N(\{q_0\}, w) = \hat \delta_N(\{q_0\}, xa) =
 \delta_N(\hat \delta_N(q_0, x), a)$. \\
 \ldots

\section{Number 6 : Exercise 2.4.1}
\begin{enumerate}

\item The automata can be represented as Figure~\ref{2.4.1.a}
\begin{figure}
% PSTricks TeX macro
% Title: W:\www\csc445\2.4.1.a.dia
% Creator: Dia v0.90
% CreationDate: Wed Sep 04 11:29:43 2002
% For: WJH3957
% \usepackage{pstricks}
% The following commands are not supported in PSTricks at present
% We define them conditionally, so when they are implemented,
% this pstricks file will use them.
\ifx\setlinejoinmode\undefined
  \newcommand{\setlinejoinmode}[1]{}
\fi
\ifx\setlinecaps\undefined
  \newcommand{\setlinecaps}[1]{}
\fi
% This way define your own fonts mapping (for example with ifthen)
\ifx\setfont\undefined
  \newcommand{\setfont}[2]{}
\fi
\pspicture(2.882165,-9.862210)(18.237770,-2.545450)
\scalebox{0.892508 -0.892508}{
\newrgbcolor{dialinecolor}{0.000000 0.000000 0.000000}
\psset{linecolor=dialinecolor}
\newrgbcolor{diafillcolor}{1.000000 1.000000 1.000000}
\psset{fillcolor=diafillcolor}
\psset{linewidth=0.100000}
\psset{linestyle=solid}
\psset{linestyle=solid}
\setlinecaps{0}
\setlinejoinmode{0}
\setlinecaps{0}
\setlinejoinmode{0}
\psset{linestyle=solid}
\newrgbcolor{diafillcolor}{1.000000 1.000000 1.000000}
\psset{fillcolor=diafillcolor}
\pscustom{
\newpath
\moveto(4.295797,6.000000)
\lineto(6.088503,6.000000)
\curveto(6.312592,6.400000)(6.384300,6.600000)(6.384300,7.000000)
\curveto(6.384300,7.400000)(6.312592,7.600000)(6.088503,8.000000)
\lineto(4.295797,8.000000)
\curveto(4.071708,7.600000)(4.000000,7.400000)(4.000000,7.000000)
\curveto(4.000000,6.600000)(4.071708,6.400000)(4.295797,6.000000)
\fill[fillstyle=solid,fillcolor=diafillcolor,linecolor=diafillcolor]}
\newrgbcolor{dialinecolor}{0.000000 0.000000 0.000000}
\psset{linecolor=dialinecolor}
\pscustom{
\newpath
\moveto(4.295797,6.000000)
\lineto(6.088503,6.000000)
\curveto(6.312592,6.400000)(6.384300,6.600000)(6.384300,7.000000)
\curveto(6.384300,7.400000)(6.312592,7.600000)(6.088503,8.000000)
\lineto(4.295797,8.000000)
\curveto(4.071708,7.600000)(4.000000,7.400000)(4.000000,7.000000)
\curveto(4.000000,6.600000)(4.071708,6.400000)(4.295797,6.000000)
\stroke}
\setfont{Courier}{0.800000}
\newrgbcolor{dialinecolor}{0.000000 0.000000 0.000000}
\psset{linecolor=dialinecolor}
\rput(5.192150,7.197980){\scalebox{1 -1}{q0}}
\psset{linewidth=0.100000}
\psset{linestyle=solid}
\psset{linestyle=solid}
\setlinecaps{0}
\setlinejoinmode{0}
\setlinecaps{0}
\setlinejoinmode{0}
\psset{linestyle=solid}
\newrgbcolor{diafillcolor}{1.000000 1.000000 1.000000}
\psset{fillcolor=diafillcolor}
\pscustom{
\newpath
\moveto(9.295797,6.000000)
\lineto(11.088503,6.000000)
\curveto(11.312592,6.400000)(11.384300,6.600000)(11.384300,7.000000)
\curveto(11.384300,7.400000)(11.312592,7.600000)(11.088503,8.000000)
\lineto(9.295797,8.000000)
\curveto(9.071708,7.600000)(9.000000,7.400000)(9.000000,7.000000)
\curveto(9.000000,6.600000)(9.071708,6.400000)(9.295797,6.000000)
\fill[fillstyle=solid,fillcolor=diafillcolor,linecolor=diafillcolor]}
\newrgbcolor{dialinecolor}{0.000000 0.000000 0.000000}
\psset{linecolor=dialinecolor}
\pscustom{
\newpath
\moveto(9.295797,6.000000)
\lineto(11.088503,6.000000)
\curveto(11.312592,6.400000)(11.384300,6.600000)(11.384300,7.000000)
\curveto(11.384300,7.400000)(11.312592,7.600000)(11.088503,8.000000)
\lineto(9.295797,8.000000)
\curveto(9.071708,7.600000)(9.000000,7.400000)(9.000000,7.000000)
\curveto(9.000000,6.600000)(9.071708,6.400000)(9.295797,6.000000)
\stroke}
\setfont{Courier}{0.800000}
\newrgbcolor{dialinecolor}{0.000000 0.000000 0.000000}
\psset{linecolor=dialinecolor}
\rput(10.192150,7.197980){\scalebox{1 -1}{q1}}
\psset{linewidth=0.100000}
\psset{linestyle=solid}
\psset{linestyle=solid}
\setlinecaps{0}
\setlinejoinmode{0}
\setlinecaps{0}
\setlinejoinmode{0}
\psset{linestyle=solid}
\newrgbcolor{diafillcolor}{1.000000 1.000000 1.000000}
\psset{fillcolor=diafillcolor}
\pscustom{
\newpath
\moveto(13.295797,3.000000)
\lineto(15.088503,3.000000)
\curveto(15.312592,3.400000)(15.384300,3.600000)(15.384300,4.000000)
\curveto(15.384300,4.400000)(15.312592,4.600000)(15.088503,5.000000)
\lineto(13.295797,5.000000)
\curveto(13.071708,4.600000)(13.000000,4.400000)(13.000000,4.000000)
\curveto(13.000000,3.600000)(13.071708,3.400000)(13.295797,3.000000)
\fill[fillstyle=solid,fillcolor=diafillcolor,linecolor=diafillcolor]}
\newrgbcolor{dialinecolor}{0.000000 0.000000 0.000000}
\psset{linecolor=dialinecolor}
\pscustom{
\newpath
\moveto(13.295797,3.000000)
\lineto(15.088503,3.000000)
\curveto(15.312592,3.400000)(15.384300,3.600000)(15.384300,4.000000)
\curveto(15.384300,4.400000)(15.312592,4.600000)(15.088503,5.000000)
\lineto(13.295797,5.000000)
\curveto(13.071708,4.600000)(13.000000,4.400000)(13.000000,4.000000)
\curveto(13.000000,3.600000)(13.071708,3.400000)(13.295797,3.000000)
\stroke}
\setfont{Courier}{0.800000}
\newrgbcolor{dialinecolor}{0.000000 0.000000 0.000000}
\psset{linecolor=dialinecolor}
\rput(14.192150,4.197980){\scalebox{1 -1}{q2}}
\psset{linewidth=0.100000}
\psset{linestyle=solid}
\psset{linestyle=solid}
\setlinecaps{0}
\setlinejoinmode{0}
\setlinecaps{0}
\setlinejoinmode{0}
\psset{linestyle=solid}
\newrgbcolor{diafillcolor}{1.000000 1.000000 1.000000}
\psset{fillcolor=diafillcolor}
\pscustom{
\newpath
\moveto(13.295797,9.000000)
\lineto(15.088503,9.000000)
\curveto(15.312592,9.400000)(15.384300,9.600000)(15.384300,10.000000)
\curveto(15.384300,10.400000)(15.312592,10.600000)(15.088503,11.000000)
\lineto(13.295797,11.000000)
\curveto(13.071708,10.600000)(13.000000,10.400000)(13.000000,10.000000)
\curveto(13.000000,9.600000)(13.071708,9.400000)(13.295797,9.000000)
\fill[fillstyle=solid,fillcolor=diafillcolor,linecolor=diafillcolor]}
\newrgbcolor{dialinecolor}{0.000000 0.000000 0.000000}
\psset{linecolor=dialinecolor}
\pscustom{
\newpath
\moveto(13.295797,9.000000)
\lineto(15.088503,9.000000)
\curveto(15.312592,9.400000)(15.384300,9.600000)(15.384300,10.000000)
\curveto(15.384300,10.400000)(15.312592,10.600000)(15.088503,11.000000)
\lineto(13.295797,11.000000)
\curveto(13.071708,10.600000)(13.000000,10.400000)(13.000000,10.000000)
\curveto(13.000000,9.600000)(13.071708,9.400000)(13.295797,9.000000)
\stroke}
\setfont{Courier}{0.800000}
\newrgbcolor{dialinecolor}{0.000000 0.000000 0.000000}
\psset{linecolor=dialinecolor}
\rput(14.192150,10.197980){\scalebox{1 -1}{q4}}
\psset{linewidth=0.100000}
\psset{linestyle=solid}
\psset{linestyle=solid}
\setlinecaps{0}
\setlinejoinmode{0}
\setlinecaps{0}
\setlinejoinmode{0}
\psset{linestyle=solid}
\newrgbcolor{diafillcolor}{1.000000 1.000000 1.000000}
\psset{fillcolor=diafillcolor}
\pscustom{
\newpath
\moveto(18.295797,9.000000)
\lineto(20.088503,9.000000)
\curveto(20.312592,9.400000)(20.384300,9.600000)(20.384300,10.000000)
\curveto(20.384300,10.400000)(20.312592,10.600000)(20.088503,11.000000)
\lineto(18.295797,11.000000)
\curveto(18.071708,10.600000)(18.000000,10.400000)(18.000000,10.000000)
\curveto(18.000000,9.600000)(18.071708,9.400000)(18.295797,9.000000)
\fill[fillstyle=solid,fillcolor=diafillcolor,linecolor=diafillcolor]}
\newrgbcolor{dialinecolor}{0.000000 0.000000 0.000000}
\psset{linecolor=dialinecolor}
\pscustom{
\newpath
\moveto(18.295797,9.000000)
\lineto(20.088503,9.000000)
\curveto(20.312592,9.400000)(20.384300,9.600000)(20.384300,10.000000)
\curveto(20.384300,10.400000)(20.312592,10.600000)(20.088503,11.000000)
\lineto(18.295797,11.000000)
\curveto(18.071708,10.600000)(18.000000,10.400000)(18.000000,10.000000)
\curveto(18.000000,9.600000)(18.071708,9.400000)(18.295797,9.000000)
\stroke}
\setfont{Courier}{0.800000}
\newrgbcolor{dialinecolor}{0.000000 0.000000 0.000000}
\psset{linecolor=dialinecolor}
\rput(19.192150,10.197980){\scalebox{1 -1}{q5}}
\psset{linewidth=0.100000}
\psset{linestyle=solid}
\psset{linestyle=solid}
\setlinecaps{0}
\setlinejoinmode{0}
\setlinecaps{0}
\setlinejoinmode{0}
\psset{linestyle=solid}
\newrgbcolor{dialinecolor}{1.000000 1.000000 1.000000}
\psset{linecolor=dialinecolor}
\pspolygon*(18.268906,3.000000)(19.971977,3.000000)(20.240883,3.300000)(20.240883,4.700000)(19.971977,5.000000)(18.268906,5.000000)(18.000000,4.700000)(18.000000,3.300000)
\newrgbcolor{dialinecolor}{0.000000 0.000000 0.000000}
\psset{linecolor=dialinecolor}
\pspolygon(18.268906,3.000000)(19.971977,3.000000)(20.240883,3.300000)(20.240883,4.700000)(19.971977,5.000000)(18.268906,5.000000)(18.000000,4.700000)(18.000000,3.300000)
\setlinecaps{0}
\setlinejoinmode{0}
\psset{linestyle=solid}
\newrgbcolor{dialinecolor}{1.000000 1.000000 1.000000}
\psset{linecolor=dialinecolor}
\pspolygon*(18.268906,3.150000)(19.971977,3.150000)(20.106430,3.300000)(20.106430,4.700000)(19.971977,4.850000)(18.268906,4.850000)(18.134453,4.700000)(18.134453,3.300000)
\newrgbcolor{dialinecolor}{0.000000 0.000000 0.000000}
\psset{linecolor=dialinecolor}
\pspolygon(18.268906,3.150000)(19.971977,3.150000)(20.106430,3.300000)(20.106430,4.700000)(19.971977,4.850000)(18.268906,4.850000)(18.134453,4.700000)(18.134453,3.300000)
\setfont{Courier}{0.800000}
\newrgbcolor{dialinecolor}{0.000000 0.000000 0.000000}
\psset{linecolor=dialinecolor}
\rput(19.120442,4.197980){\scalebox{1 -1}{q3}}
\psset{linewidth=0.100000}
\psset{linestyle=solid}
\psset{linestyle=solid}
\setlinecaps{0}
\newrgbcolor{dialinecolor}{0.000000 0.000000 0.000000}
\psset{linecolor=dialinecolor}
\psline(6.312592,7.000000)(9.000000,7.000000)
\setlinejoinmode{0}
\newrgbcolor{dialinecolor}{0.000000 0.000000 0.000000}
\psset{linecolor=dialinecolor}
\pspolygon*(8.500000,7.200000)(9.000000,7.000000)(8.500000,6.800000)
\setfont{Courier}{0.800000}
\newrgbcolor{dialinecolor}{0.000000 0.000000 0.000000}
\psset{linecolor=dialinecolor}
\rput(7.656296,6.650000){\scalebox{1 -1}{a}}
\psset{linewidth=0.100000}
\psset{linestyle=solid}
\psset{linestyle=solid}
\setlinecaps{0}
\newrgbcolor{dialinecolor}{0.000000 0.000000 0.000000}
\psset{linecolor=dialinecolor}
\psline(11.312592,7.000000)(13.050000,4.050000)
\setlinejoinmode{0}
\newrgbcolor{dialinecolor}{0.000000 0.000000 0.000000}
\psset{linecolor=dialinecolor}
\pspolygon*(12.968593,4.582328)(13.050000,4.050000)(12.623928,4.379336)
\setfont{Courier}{0.800000}
\newrgbcolor{dialinecolor}{0.000000 0.000000 0.000000}
\psset{linecolor=dialinecolor}
\rput(11.800000,4.950000){\scalebox{1 -1}{b}}
\psset{linewidth=0.100000}
\psset{linestyle=solid}
\psset{linestyle=solid}
\setlinecaps{0}
\newrgbcolor{dialinecolor}{0.000000 0.000000 0.000000}
\psset{linecolor=dialinecolor}
\psline(11.312592,7.000000)(12.887408,9.947980)
\setlinejoinmode{0}
\newrgbcolor{dialinecolor}{0.000000 0.000000 0.000000}
\psset{linecolor=dialinecolor}
\pspolygon*(12.475409,9.601199)(12.887408,9.947980)(12.828223,9.412726)
\setfont{Courier}{0.800000}
\newrgbcolor{dialinecolor}{0.000000 0.000000 0.000000}
\psset{linecolor=dialinecolor}
\rput(11.750000,9.000000){\scalebox{1 -1}{a}}
\psset{linewidth=0.100000}
\psset{linestyle=solid}
\psset{linestyle=solid}
\setlinecaps{0}
\newrgbcolor{dialinecolor}{0.000000 0.000000 0.000000}
\psset{linecolor=dialinecolor}
\psline(15.312592,10.000000)(17.900000,10.050000)
\setlinejoinmode{0}
\newrgbcolor{dialinecolor}{0.000000 0.000000 0.000000}
\psset{linecolor=dialinecolor}
\pspolygon*(17.396229,10.240302)(17.900000,10.050000)(17.403958,9.840377)
\setfont{Courier}{0.800000}
\newrgbcolor{dialinecolor}{0.000000 0.000000 0.000000}
\psset{linecolor=dialinecolor}
\rput(16.743704,9.597980){\scalebox{1 -1}{c}}
\psset{linewidth=0.100000}
\psset{linestyle=solid}
\psset{linestyle=solid}
\setlinecaps{0}
\newrgbcolor{dialinecolor}{0.000000 0.000000 0.000000}
\psset{linecolor=dialinecolor}
\psline(15.312592,4.000000)(18.000000,4.000000)
\setlinejoinmode{0}
\newrgbcolor{dialinecolor}{0.000000 0.000000 0.000000}
\psset{linecolor=dialinecolor}
\pspolygon*(17.500000,4.200000)(18.000000,4.000000)(17.500000,3.800000)
\setfont{Courier}{0.800000}
\newrgbcolor{dialinecolor}{0.000000 0.000000 0.000000}
\psset{linecolor=dialinecolor}
\rput(16.600000,3.500000){\scalebox{1 -1}{c,d}}
\psset{linewidth=0.100000}
\psset{linestyle=solid}
\psset{linestyle=solid}
\setlinecaps{0}
\newrgbcolor{dialinecolor}{0.000000 0.000000 0.000000}
\psset{linecolor=dialinecolor}
\psline(19.192150,9.000000)(19.120442,5.000000)
\setlinejoinmode{0}
\newrgbcolor{dialinecolor}{0.000000 0.000000 0.000000}
\psset{linecolor=dialinecolor}
\pspolygon*(19.329372,5.496335)(19.120442,5.000000)(18.929436,5.503504)
\setfont{Courier}{0.800000}
\newrgbcolor{dialinecolor}{0.000000 0.000000 0.000000}
\psset{linecolor=dialinecolor}
\rput(19.650000,7.150000){\scalebox{1 -1}{d}}
\psset{linewidth=0.100000}
\psset{linestyle=solid}
\psset{linestyle=solid}
\setlinejoinmode{0}
\setlinecaps{0}
\newrgbcolor{dialinecolor}{0.000000 0.000000 0.000000}
\psset{linecolor=dialinecolor}
\psline(3.300000,6.400000)(3.900000,7.000000)(3.350000,7.550000)
}\endpspicture
\caption{An automata for the strings abc, abd, and aacd}\label{2.4.1.a}
\end{figure}

\item The automata can be represented as Figure~\ref{2.4.1.b}
\begin{figure}
% PSTricks TeX macro
% Title: W:\www\csc445\2.4.1.b.dia
% Creator: Dia v0.90
% CreationDate: Wed Sep 04 11:33:11 2002
% For: WJH3957
% \usepackage{pstricks}
% The following commands are not supported in PSTricks at present
% We define them conditionally, so when they are implemented,
% this pstricks file will use them.
\ifx\setlinejoinmode\undefined
  \newcommand{\setlinejoinmode}[1]{}
\fi
\ifx\setlinecaps\undefined
  \newcommand{\setlinecaps}[1]{}
\fi
% This way define your own fonts mapping (for example with ifthen)
\ifx\setfont\undefined
  \newcommand{\setfont}[2]{}
\fi
\pspicture(7.867569,-10.564295)(23.223173,-2.820332)
\scalebox{0.956045 -0.956045}{
\newrgbcolor{dialinecolor}{0.000000 0.000000 0.000000}
\psset{linecolor=dialinecolor}
\newrgbcolor{diafillcolor}{1.000000 1.000000 1.000000}
\psset{fillcolor=diafillcolor}
\psset{linewidth=0.100000}
\psset{linestyle=solid}
\psset{linestyle=solid}
\setlinecaps{0}
\setlinejoinmode{0}
\setlinecaps{0}
\setlinejoinmode{0}
\psset{linestyle=solid}
\newrgbcolor{diafillcolor}{1.000000 1.000000 1.000000}
\psset{fillcolor=diafillcolor}
\pscustom{
\newpath
\moveto(9.295797,6.000000)
\lineto(11.088503,6.000000)
\curveto(11.312592,6.400000)(11.384300,6.600000)(11.384300,7.000000)
\curveto(11.384300,7.400000)(11.312592,7.600000)(11.088503,8.000000)
\lineto(9.295797,8.000000)
\curveto(9.071708,7.600000)(9.000000,7.400000)(9.000000,7.000000)
\curveto(9.000000,6.600000)(9.071708,6.400000)(9.295797,6.000000)
\fill[fillstyle=solid,fillcolor=diafillcolor,linecolor=diafillcolor]}
\newrgbcolor{dialinecolor}{0.000000 0.000000 0.000000}
\psset{linecolor=dialinecolor}
\pscustom{
\newpath
\moveto(9.295797,6.000000)
\lineto(11.088503,6.000000)
\curveto(11.312592,6.400000)(11.384300,6.600000)(11.384300,7.000000)
\curveto(11.384300,7.400000)(11.312592,7.600000)(11.088503,8.000000)
\lineto(9.295797,8.000000)
\curveto(9.071708,7.600000)(9.000000,7.400000)(9.000000,7.000000)
\curveto(9.000000,6.600000)(9.071708,6.400000)(9.295797,6.000000)
\stroke}
\setfont{Courier}{0.800000}
\newrgbcolor{dialinecolor}{0.000000 0.000000 0.000000}
\psset{linecolor=dialinecolor}
\rput(10.192150,7.197980){\scalebox{1 -1}{q1}}
\psset{linewidth=0.100000}
\psset{linestyle=solid}
\psset{linestyle=solid}
\setlinecaps{0}
\setlinejoinmode{0}
\setlinecaps{0}
\setlinejoinmode{0}
\psset{linestyle=solid}
\newrgbcolor{diafillcolor}{1.000000 1.000000 1.000000}
\psset{fillcolor=diafillcolor}
\pscustom{
\newpath
\moveto(13.295797,3.000000)
\lineto(15.088503,3.000000)
\curveto(15.312592,3.400000)(15.384300,3.600000)(15.384300,4.000000)
\curveto(15.384300,4.400000)(15.312592,4.600000)(15.088503,5.000000)
\lineto(13.295797,5.000000)
\curveto(13.071708,4.600000)(13.000000,4.400000)(13.000000,4.000000)
\curveto(13.000000,3.600000)(13.071708,3.400000)(13.295797,3.000000)
\fill[fillstyle=solid,fillcolor=diafillcolor,linecolor=diafillcolor]}
\newrgbcolor{dialinecolor}{0.000000 0.000000 0.000000}
\psset{linecolor=dialinecolor}
\pscustom{
\newpath
\moveto(13.295797,3.000000)
\lineto(15.088503,3.000000)
\curveto(15.312592,3.400000)(15.384300,3.600000)(15.384300,4.000000)
\curveto(15.384300,4.400000)(15.312592,4.600000)(15.088503,5.000000)
\lineto(13.295797,5.000000)
\curveto(13.071708,4.600000)(13.000000,4.400000)(13.000000,4.000000)
\curveto(13.000000,3.600000)(13.071708,3.400000)(13.295797,3.000000)
\stroke}
\setfont{Courier}{0.800000}
\newrgbcolor{dialinecolor}{0.000000 0.000000 0.000000}
\psset{linecolor=dialinecolor}
\rput(14.192150,4.197980){\scalebox{1 -1}{q2}}
\psset{linewidth=0.100000}
\psset{linestyle=solid}
\psset{linestyle=solid}
\setlinecaps{0}
\setlinejoinmode{0}
\setlinecaps{0}
\setlinejoinmode{0}
\psset{linestyle=solid}
\newrgbcolor{diafillcolor}{1.000000 1.000000 1.000000}
\psset{fillcolor=diafillcolor}
\pscustom{
\newpath
\moveto(13.295797,9.000000)
\lineto(15.088503,9.000000)
\curveto(15.312592,9.400000)(15.384300,9.600000)(15.384300,10.000000)
\curveto(15.384300,10.400000)(15.312592,10.600000)(15.088503,11.000000)
\lineto(13.295797,11.000000)
\curveto(13.071708,10.600000)(13.000000,10.400000)(13.000000,10.000000)
\curveto(13.000000,9.600000)(13.071708,9.400000)(13.295797,9.000000)
\fill[fillstyle=solid,fillcolor=diafillcolor,linecolor=diafillcolor]}
\newrgbcolor{dialinecolor}{0.000000 0.000000 0.000000}
\psset{linecolor=dialinecolor}
\pscustom{
\newpath
\moveto(13.295797,9.000000)
\lineto(15.088503,9.000000)
\curveto(15.312592,9.400000)(15.384300,9.600000)(15.384300,10.000000)
\curveto(15.384300,10.400000)(15.312592,10.600000)(15.088503,11.000000)
\lineto(13.295797,11.000000)
\curveto(13.071708,10.600000)(13.000000,10.400000)(13.000000,10.000000)
\curveto(13.000000,9.600000)(13.071708,9.400000)(13.295797,9.000000)
\stroke}
\setfont{Courier}{0.800000}
\newrgbcolor{dialinecolor}{0.000000 0.000000 0.000000}
\psset{linecolor=dialinecolor}
\rput(14.192150,10.197980){\scalebox{1 -1}{q4}}
\psset{linewidth=0.100000}
\psset{linestyle=solid}
\psset{linestyle=solid}
\setlinecaps{0}
\setlinejoinmode{0}
\setlinecaps{0}
\setlinejoinmode{0}
\psset{linestyle=solid}
\newrgbcolor{diafillcolor}{1.000000 1.000000 1.000000}
\psset{fillcolor=diafillcolor}
\pscustom{
\newpath
\moveto(18.295797,9.000000)
\lineto(20.088503,9.000000)
\curveto(20.312592,9.400000)(20.384300,9.600000)(20.384300,10.000000)
\curveto(20.384300,10.400000)(20.312592,10.600000)(20.088503,11.000000)
\lineto(18.295797,11.000000)
\curveto(18.071708,10.600000)(18.000000,10.400000)(18.000000,10.000000)
\curveto(18.000000,9.600000)(18.071708,9.400000)(18.295797,9.000000)
\fill[fillstyle=solid,fillcolor=diafillcolor,linecolor=diafillcolor]}
\newrgbcolor{dialinecolor}{0.000000 0.000000 0.000000}
\psset{linecolor=dialinecolor}
\pscustom{
\newpath
\moveto(18.295797,9.000000)
\lineto(20.088503,9.000000)
\curveto(20.312592,9.400000)(20.384300,9.600000)(20.384300,10.000000)
\curveto(20.384300,10.400000)(20.312592,10.600000)(20.088503,11.000000)
\lineto(18.295797,11.000000)
\curveto(18.071708,10.600000)(18.000000,10.400000)(18.000000,10.000000)
\curveto(18.000000,9.600000)(18.071708,9.400000)(18.295797,9.000000)
\stroke}
\setfont{Courier}{0.800000}
\newrgbcolor{dialinecolor}{0.000000 0.000000 0.000000}
\psset{linecolor=dialinecolor}
\rput(19.192150,10.197980){\scalebox{1 -1}{q5}}
\psset{linewidth=0.100000}
\psset{linestyle=solid}
\psset{linestyle=solid}
\setlinecaps{0}
\setlinejoinmode{0}
\setlinecaps{0}
\setlinejoinmode{0}
\psset{linestyle=solid}
\newrgbcolor{dialinecolor}{1.000000 1.000000 1.000000}
\psset{linecolor=dialinecolor}
\pspolygon*(22.268906,9.000000)(23.971977,9.000000)(24.240883,9.300000)(24.240883,10.700000)(23.971977,11.000000)(22.268906,11.000000)(22.000000,10.700000)(22.000000,9.300000)
\newrgbcolor{dialinecolor}{0.000000 0.000000 0.000000}
\psset{linecolor=dialinecolor}
\pspolygon(22.268906,9.000000)(23.971977,9.000000)(24.240883,9.300000)(24.240883,10.700000)(23.971977,11.000000)(22.268906,11.000000)(22.000000,10.700000)(22.000000,9.300000)
\setlinecaps{0}
\setlinejoinmode{0}
\psset{linestyle=solid}
\newrgbcolor{dialinecolor}{1.000000 1.000000 1.000000}
\psset{linecolor=dialinecolor}
\pspolygon*(22.268906,9.150000)(23.971977,9.150000)(24.106430,9.300000)(24.106430,10.700000)(23.971977,10.850000)(22.268906,10.850000)(22.134453,10.700000)(22.134453,9.300000)
\newrgbcolor{dialinecolor}{0.000000 0.000000 0.000000}
\psset{linecolor=dialinecolor}
\pspolygon(22.268906,9.150000)(23.971977,9.150000)(24.106430,9.300000)(24.106430,10.700000)(23.971977,10.850000)(22.268906,10.850000)(22.134453,10.700000)(22.134453,9.300000)
\setfont{Courier}{0.800000}
\newrgbcolor{dialinecolor}{0.000000 0.000000 0.000000}
\psset{linecolor=dialinecolor}
\rput(23.120442,10.197980){\scalebox{1 -1}{q3}}
\psset{linewidth=0.100000}
\psset{linestyle=solid}
\psset{linestyle=solid}
\setlinecaps{0}
\newrgbcolor{dialinecolor}{0.000000 0.000000 0.000000}
\psset{linecolor=dialinecolor}
\psline(11.312592,7.000000)(13.050000,4.050000)
\setlinejoinmode{0}
\newrgbcolor{dialinecolor}{0.000000 0.000000 0.000000}
\psset{linecolor=dialinecolor}
\pspolygon*(12.968593,4.582328)(13.050000,4.050000)(12.623928,4.379336)
\setfont{Courier}{0.800000}
\newrgbcolor{dialinecolor}{0.000000 0.000000 0.000000}
\psset{linecolor=dialinecolor}
\rput(11.800000,4.950000){\scalebox{1 -1}{0}}
\psset{linewidth=0.100000}
\psset{linestyle=solid}
\psset{linestyle=solid}
\setlinecaps{0}
\newrgbcolor{dialinecolor}{0.000000 0.000000 0.000000}
\psset{linecolor=dialinecolor}
\psline(11.312592,7.000000)(12.887408,9.947980)
\setlinejoinmode{0}
\newrgbcolor{dialinecolor}{0.000000 0.000000 0.000000}
\psset{linecolor=dialinecolor}
\pspolygon*(12.475409,9.601199)(12.887408,9.947980)(12.828223,9.412726)
\setfont{Courier}{0.800000}
\newrgbcolor{dialinecolor}{0.000000 0.000000 0.000000}
\psset{linecolor=dialinecolor}
\rput(11.750000,9.000000){\scalebox{1 -1}{1}}
\psset{linewidth=0.100000}
\psset{linestyle=solid}
\psset{linestyle=solid}
\setlinecaps{0}
\newrgbcolor{dialinecolor}{0.000000 0.000000 0.000000}
\psset{linecolor=dialinecolor}
\psline(15.312592,10.000000)(17.900000,10.050000)
\setlinejoinmode{0}
\newrgbcolor{dialinecolor}{0.000000 0.000000 0.000000}
\psset{linecolor=dialinecolor}
\pspolygon*(17.396229,10.240302)(17.900000,10.050000)(17.403958,9.840377)
\setfont{Courier}{0.800000}
\newrgbcolor{dialinecolor}{0.000000 0.000000 0.000000}
\psset{linecolor=dialinecolor}
\rput(16.743704,9.597980){\scalebox{1 -1}{0}}
\psset{linewidth=0.100000}
\psset{linestyle=solid}
\psset{linestyle=solid}
\setlinecaps{0}
\newrgbcolor{dialinecolor}{0.000000 0.000000 0.000000}
\psset{linecolor=dialinecolor}
\psline(14.192150,5.000000)(14.192150,9.000000)
\setlinejoinmode{0}
\newrgbcolor{dialinecolor}{0.000000 0.000000 0.000000}
\psset{linecolor=dialinecolor}
\pspolygon*(13.992150,8.500000)(14.192150,9.000000)(14.392150,8.500000)
\setfont{Courier}{0.800000}
\newrgbcolor{dialinecolor}{0.000000 0.000000 0.000000}
\psset{linecolor=dialinecolor}
\rput(14.900000,7.100000){\scalebox{1 -1}{1}}
\psset{linewidth=0.100000}
\psset{linestyle=solid}
\psset{linestyle=solid}
\setlinecaps{0}
\newrgbcolor{dialinecolor}{0.000000 0.000000 0.000000}
\psset{linecolor=dialinecolor}
\psline(15.312592,4.000000)(19.192150,9.000000)
\setlinejoinmode{0}
\newrgbcolor{dialinecolor}{0.000000 0.000000 0.000000}
\psset{linecolor=dialinecolor}
\pspolygon*(18.727626,8.727571)(19.192150,9.000000)(19.043652,8.482363)
\setfont{Courier}{0.800000}
\newrgbcolor{dialinecolor}{0.000000 0.000000 0.000000}
\psset{linecolor=dialinecolor}
\rput(18.400000,6.950000){\scalebox{1 -1}{1}}
\psset{linewidth=0.100000}
\psset{linestyle=solid}
\psset{linestyle=solid}
\setlinejoinmode{0}
\setlinecaps{0}
\newrgbcolor{dialinecolor}{0.000000 0.000000 0.000000}
\psset{linecolor=dialinecolor}
\psline(8.300000,6.450000)(8.900000,7.050000)(8.350000,7.600000)
\psset{linewidth=0.100000}
\psset{linestyle=solid}
\psset{linestyle=solid}
\setlinecaps{0}
\newrgbcolor{dialinecolor}{0.000000 0.000000 0.000000}
\psset{linecolor=dialinecolor}
\psline(20.312592,10.000000)(22.000000,10.000000)
\setlinejoinmode{0}
\newrgbcolor{dialinecolor}{0.000000 0.000000 0.000000}
\psset{linecolor=dialinecolor}
\pspolygon*(21.500000,10.200000)(22.000000,10.000000)(21.500000,9.800000)
\setfont{Courier}{0.800000}
\newrgbcolor{dialinecolor}{0.000000 0.000000 0.000000}
\psset{linecolor=dialinecolor}
\rput(21.050000,9.550000){\scalebox{1 -1}{1}}
}\endpspicture
\caption{An automata for the strings 0101, 011, and 101}\label{2.4.1.b}
\end{figure}

\item The automata can be represented as Figure~\ref{2.4.1.c}
\begin{figure}
% PSTricks TeX macro
% Title: W:\www\csc445\2.4.1.c.dia
% Creator: Dia v0.90
% CreationDate: Wed Sep 04 11:46:17 2002
% For: WJH3957
% \usepackage{pstricks}
% The following commands are not supported in PSTricks at present
% We define them conditionally, so when they are implemented,
% this pstricks file will use them.
\ifx\setlinejoinmode\undefined
  \newcommand{\setlinejoinmode}[1]{}
\fi
\ifx\setlinecaps\undefined
  \newcommand{\setlinecaps}[1]{}
\fi
% This way define your own fonts mapping (for example with ifthen)
\ifx\setfont\undefined
  \newcommand{\setfont}[2]{}
\fi
\pspicture(9.674601,-14.166344)(25.030206,-2.292479)
\scalebox{1.175630 -1.175630}{
\newrgbcolor{dialinecolor}{0.000000 0.000000 0.000000}
\psset{linecolor=dialinecolor}
\newrgbcolor{diafillcolor}{1.000000 1.000000 1.000000}
\psset{fillcolor=diafillcolor}
\psset{linewidth=0.100000}
\psset{linestyle=solid}
\psset{linestyle=solid}
\setlinecaps{0}
\setlinejoinmode{0}
\setlinecaps{0}
\setlinejoinmode{0}
\psset{linestyle=solid}
\newrgbcolor{diafillcolor}{1.000000 1.000000 1.000000}
\psset{fillcolor=diafillcolor}
\pscustom{
\newpath
\moveto(9.295797,6.000000)
\lineto(11.088503,6.000000)
\curveto(11.312592,6.400000)(11.384300,6.600000)(11.384300,7.000000)
\curveto(11.384300,7.400000)(11.312592,7.600000)(11.088503,8.000000)
\lineto(9.295797,8.000000)
\curveto(9.071708,7.600000)(9.000000,7.400000)(9.000000,7.000000)
\curveto(9.000000,6.600000)(9.071708,6.400000)(9.295797,6.000000)
\fill[fillstyle=solid,fillcolor=diafillcolor,linecolor=diafillcolor]}
\newrgbcolor{dialinecolor}{0.000000 0.000000 0.000000}
\psset{linecolor=dialinecolor}
\pscustom{
\newpath
\moveto(9.295797,6.000000)
\lineto(11.088503,6.000000)
\curveto(11.312592,6.400000)(11.384300,6.600000)(11.384300,7.000000)
\curveto(11.384300,7.400000)(11.312592,7.600000)(11.088503,8.000000)
\lineto(9.295797,8.000000)
\curveto(9.071708,7.600000)(9.000000,7.400000)(9.000000,7.000000)
\curveto(9.000000,6.600000)(9.071708,6.400000)(9.295797,6.000000)
\stroke}
\setfont{Courier}{0.800000}
\newrgbcolor{dialinecolor}{0.000000 0.000000 0.000000}
\psset{linecolor=dialinecolor}
\rput(10.192150,7.197980){\scalebox{1 -1}{q1}}
\psset{linewidth=0.100000}
\psset{linestyle=solid}
\psset{linestyle=solid}
\setlinecaps{0}
\setlinejoinmode{0}
\setlinecaps{0}
\setlinejoinmode{0}
\psset{linestyle=solid}
\newrgbcolor{diafillcolor}{1.000000 1.000000 1.000000}
\psset{fillcolor=diafillcolor}
\pscustom{
\newpath
\moveto(14.295797,2.000000)
\lineto(16.088503,2.000000)
\curveto(16.312592,2.400000)(16.384300,2.600000)(16.384300,3.000000)
\curveto(16.384300,3.400000)(16.312592,3.600000)(16.088503,4.000000)
\lineto(14.295797,4.000000)
\curveto(14.071708,3.600000)(14.000000,3.400000)(14.000000,3.000000)
\curveto(14.000000,2.600000)(14.071708,2.400000)(14.295797,2.000000)
\fill[fillstyle=solid,fillcolor=diafillcolor,linecolor=diafillcolor]}
\newrgbcolor{dialinecolor}{0.000000 0.000000 0.000000}
\psset{linecolor=dialinecolor}
\pscustom{
\newpath
\moveto(14.295797,2.000000)
\lineto(16.088503,2.000000)
\curveto(16.312592,2.400000)(16.384300,2.600000)(16.384300,3.000000)
\curveto(16.384300,3.400000)(16.312592,3.600000)(16.088503,4.000000)
\lineto(14.295797,4.000000)
\curveto(14.071708,3.600000)(14.000000,3.400000)(14.000000,3.000000)
\curveto(14.000000,2.600000)(14.071708,2.400000)(14.295797,2.000000)
\stroke}
\setfont{Courier}{0.800000}
\newrgbcolor{dialinecolor}{0.000000 0.000000 0.000000}
\psset{linecolor=dialinecolor}
\rput(15.192150,3.197980){\scalebox{1 -1}{q2}}
\psset{linewidth=0.100000}
\psset{linestyle=solid}
\psset{linestyle=solid}
\setlinecaps{0}
\setlinejoinmode{0}
\setlinecaps{0}
\setlinejoinmode{0}
\psset{linestyle=solid}
\newrgbcolor{diafillcolor}{1.000000 1.000000 1.000000}
\psset{fillcolor=diafillcolor}
\pscustom{
\newpath
\moveto(14.295797,10.000000)
\lineto(16.088503,10.000000)
\curveto(16.312592,10.400000)(16.384300,10.600000)(16.384300,11.000000)
\curveto(16.384300,11.400000)(16.312592,11.600000)(16.088503,12.000000)
\lineto(14.295797,12.000000)
\curveto(14.071708,11.600000)(14.000000,11.400000)(14.000000,11.000000)
\curveto(14.000000,10.600000)(14.071708,10.400000)(14.295797,10.000000)
\fill[fillstyle=solid,fillcolor=diafillcolor,linecolor=diafillcolor]}
\newrgbcolor{dialinecolor}{0.000000 0.000000 0.000000}
\psset{linecolor=dialinecolor}
\pscustom{
\newpath
\moveto(14.295797,10.000000)
\lineto(16.088503,10.000000)
\curveto(16.312592,10.400000)(16.384300,10.600000)(16.384300,11.000000)
\curveto(16.384300,11.400000)(16.312592,11.600000)(16.088503,12.000000)
\lineto(14.295797,12.000000)
\curveto(14.071708,11.600000)(14.000000,11.400000)(14.000000,11.000000)
\curveto(14.000000,10.600000)(14.071708,10.400000)(14.295797,10.000000)
\stroke}
\setfont{Courier}{0.800000}
\newrgbcolor{dialinecolor}{0.000000 0.000000 0.000000}
\psset{linecolor=dialinecolor}
\rput(15.192150,11.197980){\scalebox{1 -1}{q4}}
\psset{linewidth=0.100000}
\psset{linestyle=solid}
\psset{linestyle=solid}
\setlinecaps{0}
\setlinejoinmode{0}
\setlinecaps{0}
\setlinejoinmode{0}
\psset{linestyle=solid}
\newrgbcolor{dialinecolor}{1.000000 1.000000 1.000000}
\psset{linecolor=dialinecolor}
\pspolygon*(19.268906,6.000000)(20.971977,6.000000)(21.240883,6.300000)(21.240883,7.700000)(20.971977,8.000000)(19.268906,8.000000)(19.000000,7.700000)(19.000000,6.300000)
\newrgbcolor{dialinecolor}{0.000000 0.000000 0.000000}
\psset{linecolor=dialinecolor}
\pspolygon(19.268906,6.000000)(20.971977,6.000000)(21.240883,6.300000)(21.240883,7.700000)(20.971977,8.000000)(19.268906,8.000000)(19.000000,7.700000)(19.000000,6.300000)
\setlinecaps{0}
\setlinejoinmode{0}
\psset{linestyle=solid}
\newrgbcolor{dialinecolor}{1.000000 1.000000 1.000000}
\psset{linecolor=dialinecolor}
\pspolygon*(19.268906,6.150000)(20.971977,6.150000)(21.106430,6.300000)(21.106430,7.700000)(20.971977,7.850000)(19.268906,7.850000)(19.134453,7.700000)(19.134453,6.300000)
\newrgbcolor{dialinecolor}{0.000000 0.000000 0.000000}
\psset{linecolor=dialinecolor}
\pspolygon(19.268906,6.150000)(20.971977,6.150000)(21.106430,6.300000)(21.106430,7.700000)(20.971977,7.850000)(19.268906,7.850000)(19.134453,7.700000)(19.134453,6.300000)
\setfont{Courier}{0.800000}
\newrgbcolor{dialinecolor}{0.000000 0.000000 0.000000}
\psset{linecolor=dialinecolor}
\rput(20.120442,7.197980){\scalebox{1 -1}{q5}}
\psset{linewidth=0.100000}
\psset{linestyle=solid}
\psset{linestyle=solid}
\setlinecaps{0}
\newrgbcolor{dialinecolor}{0.000000 0.000000 0.000000}
\psset{linecolor=dialinecolor}
\psline(10.192150,6.000000)(14.000000,3.000000)
\setlinejoinmode{0}
\newrgbcolor{dialinecolor}{0.000000 0.000000 0.000000}
\psset{linecolor=dialinecolor}
\pspolygon*(13.731019,3.466529)(14.000000,3.000000)(13.483477,3.152327)
\setfont{Courier}{0.800000}
\newrgbcolor{dialinecolor}{0.000000 0.000000 0.000000}
\psset{linecolor=dialinecolor}
\rput(11.900099,4.150000){\scalebox{1 -1}{a}}
\psset{linewidth=0.100000}
\psset{linestyle=solid}
\psset{linestyle=solid}
\setlinecaps{0}
\newrgbcolor{dialinecolor}{0.000000 0.000000 0.000000}
\psset{linecolor=dialinecolor}
\psline(10.192150,8.000000)(14.000000,11.000000)
\setlinejoinmode{0}
\newrgbcolor{dialinecolor}{0.000000 0.000000 0.000000}
\psset{linecolor=dialinecolor}
\pspolygon*(13.483477,10.847673)(14.000000,11.000000)(13.731019,10.533471)
\setfont{Courier}{0.800000}
\newrgbcolor{dialinecolor}{0.000000 0.000000 0.000000}
\psset{linecolor=dialinecolor}
\rput(11.850000,10.100000){\scalebox{1 -1}{c}}
\psset{linewidth=0.100000}
\psset{linestyle=solid}
\psset{linestyle=solid}
\setlinecaps{0}
\newrgbcolor{dialinecolor}{0.000000 0.000000 0.000000}
\psset{linecolor=dialinecolor}
\psline(16.312592,11.000000)(20.120442,8.000000)
\setlinejoinmode{0}
\newrgbcolor{dialinecolor}{0.000000 0.000000 0.000000}
\psset{linecolor=dialinecolor}
\pspolygon*(19.851461,8.466529)(20.120442,8.000000)(19.603918,8.152327)
\setfont{Courier}{0.800000}
\newrgbcolor{dialinecolor}{0.000000 0.000000 0.000000}
\psset{linecolor=dialinecolor}
\rput(12.700099,6.650000){\scalebox{1 -1}{b}}
\psset{linewidth=0.100000}
\psset{linestyle=solid}
\psset{linestyle=solid}
\setlinecaps{0}
\newrgbcolor{dialinecolor}{0.000000 0.000000 0.000000}
\psset{linecolor=dialinecolor}
\psline(16.312592,3.000000)(20.120442,6.000000)
\setlinejoinmode{0}
\newrgbcolor{dialinecolor}{0.000000 0.000000 0.000000}
\psset{linecolor=dialinecolor}
\pspolygon*(19.603918,5.847673)(20.120442,6.000000)(19.851461,5.533471)
\setfont{Courier}{0.800000}
\newrgbcolor{dialinecolor}{0.000000 0.000000 0.000000}
\psset{linecolor=dialinecolor}
\rput(17.650000,6.700000){\scalebox{1 -1}{c}}
\psset{linewidth=0.100000}
\psset{linestyle=solid}
\psset{linestyle=solid}
\setlinejoinmode{0}
\setlinecaps{0}
\newrgbcolor{dialinecolor}{0.000000 0.000000 0.000000}
\psset{linecolor=dialinecolor}
\psline(8.300000,6.450000)(8.900000,7.050000)(8.350000,7.600000)
\psset{linewidth=0.100000}
\psset{linestyle=solid}
\psset{linestyle=solid}
\setlinecaps{0}
\newrgbcolor{dialinecolor}{0.000000 0.000000 0.000000}
\psset{linecolor=dialinecolor}
\psline(16.312592,7.000000)(19.000000,7.000000)
\setlinejoinmode{0}
\newrgbcolor{dialinecolor}{0.000000 0.000000 0.000000}
\psset{linecolor=dialinecolor}
\pspolygon*(18.500000,7.200000)(19.000000,7.000000)(18.500000,6.800000)
\psset{linewidth=0.100000}
\psset{linestyle=solid}
\psset{linestyle=solid}
\setlinecaps{0}
\setlinejoinmode{0}
\setlinecaps{0}
\setlinejoinmode{0}
\psset{linestyle=solid}
\newrgbcolor{diafillcolor}{1.000000 1.000000 1.000000}
\psset{fillcolor=diafillcolor}
\pscustom{
\newpath
\moveto(14.295797,6.000000)
\lineto(16.088503,6.000000)
\curveto(16.312592,6.400000)(16.384300,6.600000)(16.384300,7.000000)
\curveto(16.384300,7.400000)(16.312592,7.600000)(16.088503,8.000000)
\lineto(14.295797,8.000000)
\curveto(14.071708,7.600000)(14.000000,7.400000)(14.000000,7.000000)
\curveto(14.000000,6.600000)(14.071708,6.400000)(14.295797,6.000000)
\fill[fillstyle=solid,fillcolor=diafillcolor,linecolor=diafillcolor]}
\newrgbcolor{dialinecolor}{0.000000 0.000000 0.000000}
\psset{linecolor=dialinecolor}
\pscustom{
\newpath
\moveto(14.295797,6.000000)
\lineto(16.088503,6.000000)
\curveto(16.312592,6.400000)(16.384300,6.600000)(16.384300,7.000000)
\curveto(16.384300,7.400000)(16.312592,7.600000)(16.088503,8.000000)
\lineto(14.295797,8.000000)
\curveto(14.071708,7.600000)(14.000000,7.400000)(14.000000,7.000000)
\curveto(14.000000,6.600000)(14.071708,6.400000)(14.295797,6.000000)
\stroke}
\setfont{Courier}{0.800000}
\newrgbcolor{dialinecolor}{0.000000 0.000000 0.000000}
\psset{linecolor=dialinecolor}
\rput(15.192150,7.197980){\scalebox{1 -1}{q3}}
\psset{linewidth=0.100000}
\psset{linestyle=solid}
\psset{linestyle=solid}
\setlinecaps{0}
\newrgbcolor{dialinecolor}{0.000000 0.000000 0.000000}
\psset{linecolor=dialinecolor}
\psline(11.312592,7.000000)(14.000000,7.000000)
\setlinejoinmode{0}
\newrgbcolor{dialinecolor}{0.000000 0.000000 0.000000}
\psset{linecolor=dialinecolor}
\pspolygon*(13.500000,7.200000)(14.000000,7.000000)(13.500000,6.800000)
\setfont{Courier}{0.800000}
\newrgbcolor{dialinecolor}{0.000000 0.000000 0.000000}
\psset{linecolor=dialinecolor}
\rput(18.500099,10.150000){\scalebox{1 -1}{a}}
\setfont{Courier}{0.800000}
\newrgbcolor{dialinecolor}{0.000000 0.000000 0.000000}
\psset{linecolor=dialinecolor}
\rput(18.400099,4.150000){\scalebox{1 -1}{b}}
}\endpspicture
\caption{An automata for the strings ab, bc, ca}\label{2.4.1.c}
\end{figure}

\end{enumerate}

\section{Number 7 : Exercise 2.5.1}
\begin{enumerate}
\item The $\epsilon$-closures are: \\
 $\epsilon_c(p) = \{p\}$ \\
 $\epsilon_c(q) = \{p, q\}$ \\
 $\epsilon_c(r) = \{p, q, r\}$

\item All strings of length 3 or less $= \{w | |w| \leq 3; (count_b(w)
  \geq 2$ and/or $count_c(w) \geq 1)\}$

\item As a DFA:
\begin{table}
\begin{tabular}{c || c | c | c}
State                   &a              &b         &c \\
\hline\hline
$\rightarrow \{p\}$     & $\{p\}$       & $\{p, q\}$    & $\{p, q, r\}$ \\
$\{p, q\}$              & $\{p, q\}$    & $\{p, q, r\}$ & $\{p, q, r\}$ \\
$\{p, q, r\}$           & $\{p, q, r\}$ & $\{p, q, r\}$ & $\{p, q, r\}$
\end{tabular}
\caption{Transitions for $\delta_D$}\label{2.5.1.c}
\end{table}

$F = \{\{p, q, r\}\}$

\end{enumerate}
  
\section{Number 8 : Exercise 2.5.2}
\begin{enumerate}
\item The $\epsilon$-closures are: \\
 $\epsilon_c(p) = \{p, q, r\}$ \\
 $\epsilon_c(q) = \{q\}$ \\
 $\epsilon_c(p) = \{r\}$ \\

\item All strings of length 3 or less $= \{x_1x_2x_3 | x_1 \in
  \{a,b,c\}; x_2 \in \{a,c\}; x_3 \in \{a,b,c,\epsilon\}\} \cup
  \{abb, abc, bb, a, b, c, \epsilon\}$

\item As a DFA:
\begin{table}
\begin{tabular}{c || c | c | c}
State                        &a              &b            &c \\
\hline\hline
$\rightarrow \{p, q, r\}$    & $\{p\}$       & $\{q, r\}$  & $\{p, q, r\}$ \\
$\{p\}$                      & $\emptyset$   & $\{q\}$     & $\{r\}$       \\
$\{q, r\}$                   & $\{p, q, r\}$ & $\{r\}$     & $\{p, q, r\}$ \\
$\{q\}$                      & $\{p, q, r\}$ & $\{r\}$     & $\{p, q, r\}$ \\
$\{r\}$                      & $\emptyset$   & $\emptyset$ & $\emptyset$
\end{tabular}
\caption{Transitions for $\delta_D$}\label{2.5.2.c}
\end{table}

$F = \{\{p, q, r\}, \{q, r\}, \{r\}\}$


\end{enumerate}

\end{document}
