\documentclass[12pt,a4paper,twoside]{article}  % Comments after  % are ignored
\usepackage{amsmath,amssymb,amsfonts}          % Typical maths resource packages
\usepackage{hyperref}                          % For creating hyperlinks in cross references

\pagestyle{headings}         % Option to put page headers
                             % Needed \documentclass[a4paper,twoside]{article}

%\textwidth 17.5cm
%\rightmargin 1in

%\topmargin -1cm
%\parindent 0cm
%\textheight 24cm
%\parskip 1mm

%\newtheorem{definition}[theorem]{Definition}

%\def\R{\mathbb{ R}}
%\def\S{\mathbb{ S}}

\author{Will Holcomb\\
  \small{CSC203 - Proactical and Professional Issues in Computer Science}}
\title{Commentary on the ACM Code of Ethics}
\date{\small\it August 23, 2002}

\begin{document}
\maketitle
\begin{abstract}
A report summarizing the most important aspects of the ACM Software
Engineering Code of Ethics and Professional Practice. \cite{acm} Along
with a critique as to aspects that seem unreasonable and an assessment
of the code's applicability to computer science students today.
\end{abstract}

\section{Introduction}
The Association for Computing Machinery (ACM) is one of the primary
professional organizations for computer scientists. They publish a
code of ethics \cite{acm} which they expect all members to abide
by. It is broken into four main sections.

\begin{enumerate}
\item Fundamental ethical considerations
\item Professional conduct
\item Leadership guidelines
\item Compliance principles
\end{enumerate}

\section{The Principle}
The concept of a code of ethics is certainly an admirable one. One of
the most disturbing developments accompanying the rise of
industrialization and technology has been the way that the
accompanying stratification of society has allowed individuals to lose
touch with the consequences of their actions.

Pragmatically, it is not particularly meaningful. Research into moral
development \cite{kohlberg} suggests that individuals who would be
operating at the level outlined in this document (Kohlberg 4-5) will
more than likely already be operating from an internalized moral model
and the majority of people operating from a less developed stage will
likely not internalize the document.

I suppose it is useful though in situations like this to help people
have a larger context for what their actions are. Many people may not
have considered that their work as programmers may have far reaching
effects on the world at large.

\section{Specificity}
The document is too specific at times. With passages like:
\begin{quote}
Commitment to ethical professional conduct is expected of every member
(voting members, associate members, and student members) of the
Association for Computing Machinery (ACM).
\end{quote}
and
\begin{quote}
This principle prohibits use of computing technology in ways that
result in harm to any of the following: users, the general public,
employees, employers.
\end{quote}
It reads like an algorithm. This level of specificity is dangerous in
a field which is highly dependent on individual circumstances for
reasoning. I think that it would be improved in some places with more
general language. In many ways it almost sounds like it had multiple
authors since some sections are decidedly general:
\begin{quote}
This principle concerning the quality of life of all people affirms an
obligation to protect fundamental human rights and to respect the
diversity of all cultures. An essential aim of computing professionals
is to minimize negative consequences of computing systems, including
threats to health and safety.
\end{quote}

I like it though when they give specific examples which might actually
happen in the work environment. I think that helps tie in the abstract
principles to the real world in a way that may help people recognize
the authors intent better. In particular I like the focus on
recognizing the repercussions of actions and the awareness that
upholding an ethical code may have undesirable personal consequences.

\section{Intellectual Property}
There are a variety of issues in the modern world where ethical
concerns over intellectual property conflict with other concerns. I
think that a blanket statement agreeing to respect all intellectual
property laws regardless of circumstance in naive.

As a computer scientist though I recognize that pretty much all of the
work that I produce in my life will be intellectual. I recognize that
the system of laws we have protects me and I agree with Kant on the
importance of categorical imperatives in determining actions that are
sustainable within a society.

\section{Excellence}
From the section on professional behavior I could not agree more with
the statement that ``Excellence is perhaps the most important
obligation of a professional''. From both working professionally and
on projects with my peers it is extremely frustrating to see quality
of work being sacrificed continually to avoid work. In the marketplace
it was more reasonable since there were always encroaching deadlines,
but at school I think that excellence and professionalism should get
much more emphasis.

In a way I am uncomfortable with my position on this issue. I
recognize that I enjoy learning and it is something that comes easily
to me. Designing software is something that I would continue to do
even if I had no monetary motivation. Given that I enjoy it, how fair
is it of me to require the same standards of those who do this as a
job rather than a passion.

I could go on at length about the ethical implications of professional
competence on ethical ramifications of work, but since I am already
past my limit I will refrain \ldots

\section{Summary}
All in all I like the document and agree with it. I like the
Thoreauesque civil disobedience stuff in the professional practice
section. Having worked in industry some I know how the bottom line can
sometimes cloud the larger picture. In particular I liked the
statement that:
\begin{quote}
If a member does not follow this code by engaging in gross misconduct,
membership in ACM may be terminated.
\end{quote}

While I recognize the limitation of overly idealized positions I
think that having individuals and organizations stating and upholding
a set of ideals is useful to the overall ethical development of society.

\begin{thebibliography}{99}
\bibitem{acm} Association for Computing Machinery. {\bf ACM Code of
Ethics and Professional Conduct}, Task Force for the Revision of the
ACM Code of Ethics and Professional Conduct. 1992.

\bibitem{kohlberg}Kohlberg, Lawrence. {\bf The Meaning and Measurement
of Moral Development}. 1979 Heinz Werner Lecture Series.
\end{thebibliography}

\end{document}
